\subsubsection{ Mean Field Models (or similar) in Economic Research}
\begin{itemize}
    \item Moll, Lucas
    \item Models for Heterogeneous Agents
    \item Gabaix, Dynamics of Inequality, 2016 - Sharp rise in inequality not modelled accurately by standard random growth models. The paper suggests some changes to standard models that address this issue: type dependence and scale dependence in wealth accumulation.
    \item TAPPING INTO TALENT - Model for economy where agents can chose between working on industry or working as a researcher. Has an educational component - agents can only chose to become researchers if they pursue an PhD. Pursuing a PhD has "innate talent" and economic restrictions. Researcher's wage are calculated by solving for the market equilibrium in a free market of ideas. Model is calibrated to census data from Denmark and used to study counterfactual policy exercises.
    \item Non market interactions - Scheinkman - check for more recent references
\end{itemize}

\subsubsection{Econometric Research on Education}
\begin{itemize}
    \item Articles with stylized facts - Tapping Into Talent
    \item articles that measure interesting behaviour that we hope the model can replicate.
    \item articles that argue in favor of the importance of this research
    \item Growth models review - specially Lucas and Ayiagari
\end{itemize}
