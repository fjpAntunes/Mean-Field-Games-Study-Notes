Consider an economy with $N$ agents, evolving in the time interval $[0,T]$.
Each agent is described at time $t$ by his wealth $A_t$ and his skill level $H_t$,
with $(A_t, H_t) \in \Omega = \mathbb{R} \times \mathbb{R}^+$ for every $t \in [0,T]$.
Agents control their consumption $c_t \in \mathbb{R}^+$, and the proportion of time they dedicate to working $u_t \in [0,1]$. 
We assume that the proportion $1 - u_t$ of time not used to working is used to improve their skill level. 

Each agent face the following optimization problem:
\begin{equation}
\begin{cases}
        \sup\limits_{(u,c) \in \mathcal{U} \times \mathcal{C}}\mathbb{E} [ \int_0^T f_c(c_s) + f_u(u_s) ds + Q(A_T) ], \text{ s.t.}\\
        d A_t = \left[ (\bar r_t - \delta) A_t + \bar w_t H_t u_t - c_t  \right] dt + \sigma_a A_t d W^a_t.
        d H_t = H^\xi_t g(1 - u_t) dt + \sigma_h H_t d W^h_t,\\
\end{cases}
\end{equation}
The term $f_c(c_s)$ is a running utility for consumption, whereas $f_u(u_s)$ is a running utility for proportion of time spent working.
The term $Q(A_T)$ represents utility over terminal wealth.

The agent's skill level $H_t$ evolves through time with a drift term $H^\xi_t g(1 - u_t) dt$ and a noise term $\sigma_h H_t dW^h_t$.
The function $g$ represents the effectiveness of time $1 - u_t$ employed in improving skill level,
 whereas the term $H_t^\xi$ captures the impact of current skill level in effectiveness of effort.
 For $\xi > 0$, higher skill levels lead to higher effectiveness.
  However, for $\xi < 0$, higher skill levels lead to lower effectiveness.
  As for $\xi = 0$, the skill level is indifferent to effectiveness.

The agent's wealth $A_t$ evolves with a drift term $\left[ (\bar r_t - \delta) A_t + \bar w_t H_t u_t - c_t  \right] dt $ and a noise term $\sigma_a A_t d W^a_t$.
The drift term is composed of interest returns over current wealth, wages proportional to time dedicated to working and a reduction on wealth due to consumption given by $c_t$.
The interest returns are described by $(\bar r_t - \delta) A_t$, where $\delta$ is a depreciation rate and $\bar r_t$ is the interest rate of the economy, to be determined endogenously through mean field interactions.
The wages are given by $\bar w_t H_t u_t$, where $H_t$ is the agent's skill level, $u_t$ is the proportion of time dedicated to work and $\bar w_t$ is the wage rate per time per skill of the economy, which also will be determined endogenously through mean field interactions.

The noise terms $\sigma_h H_t dW^h_t$ and $\sigma_a A_t dW^a_t$ are driven by independent brownian motions $(W^h_t)_{t \in [0,T]}$ and $W^a_t$.
We consider that each agent has independent noise.

\textcolor{red}{elaborate on the aggregate quantities description}
Interaction between agents is mediated through $\bar r_t$ and $\bar w_t$.
These values depend on aggreates of the agent's states and on the production function of the economy as folows:
Suppose that the economy has a Cobb-Douglas production \cite{find a citation for cobb douglas} function $F$ which depends on average wealth $\bar a_t$ 
and on the effective supply of skilled labor $\bar h_t^{eff}$.
At time $t$, the agents are distributed over $\Omega$ with a measure $\mu^N_t \in \mathcal{P}(\Omega)$ given by
\[
\mu^N_t = \frac{1}{N} \sum_{i = 1}^N \delta_{(A^i_t, H^i_t)}.
\]
Average wealth is then given by
\[
\bar a_t = \int_\Omega \, a \, d\mu^N_t (a,h),
\]
whereas the effective supply of skilled labor is given by
\[
\bar h_t^{eff} = \int_\Omega \, u(a,h) h \, d\mu^N_t (a,h).
\]
The interpretation for average wealth is straightforward.
The effective supply of skilled labor is a weighted average of skill over the population, where the weights $u(a,h)$ are the proportion of time dedicated to work as a function of the state $(a,h)$.
It is clear that $\bar h^{eff} \leq \int_\Omega  h \, d\mu^N_t (a,h) = \bar h$, where equality holds if every agent dedicates his whole time to work.

At economic equilibrium, the interest rate and the wage rate are equal to their marginal increase in production \cite{find a citation for this}, that is
\[
\bar r_t= \partial_a F(\bar a_t, \bar h_t^{eff}), \quad \bar w_t= \partial_h F(\bar a_t, \bar h_t^{eff}).
\]
For a production function $F(a,h) = C a^\beta h^{1 - \beta}$, $0 \leq \beta \leq 1$, we have
\[
\bar r_t = C \beta {\bar a}^{\beta - 1} ({\bar h^{eff}})^{1 - beta}, \quad  \bar w_t = C (1 - \beta) {\bar a}^{beta} ({\bar h^{eff}})^{ - beta}.
\]


\textcolor{red}{analyze monotonicity and convexity wrt. aggregate quantities}


Solutions for \eqref{education_model:mfg_analytic_system} describe Nash equilibria for the mean-field limit of the game.
The function $V: [0,T] \times \Omega \mapsto \mathbb{R}$ is the value function for the game at the Nash equilibrium,
whereas the probability measure flow $\mu: [0,T] \times \Omega \mapsto \mathbb{R}$ describe the time evolution of the  population density over the state space $\Omega$.
\begin{equation}\label{education_model:mfg_analytic_system}
    \begin{cases}
        \partial_t V + (\bar r  - \delta) a \partial_a V + \HH_u  + \HH_c + \frac{1}{2} \sigma_a^2 a^2 \partial_{aa} V + \frac{1}{2} \sigma^2_h h^2 \partial_{hh} V = 0,\\
        \partial_t \mu + \partial_a \left( \left[ (\bar r - \delta) a + \partial_p \HH_u + \partial_p \HH_c \right] \mu \right)  + \partial_h \left( \partial_q \HH_u\, \mu\right)  - \frac{1}{2} \sigma_a^2 \partial_{aa} (a^2\mu) - \frac{1}{2} \sigma^2_h \partial_{hh} (h^2\mu) = 0,\\
        \mu(0,a,h) = \mu_0,\quad V(T,a,h) = Q(a)
    \end{cases}
\end{equation}
where
\begin{equation}
    \begin{cases}
        \HH_u(h,p,q) = \sup\limits_{u} \left \{ h^\xi \, g(1 - u)\, q + h u \, \bar w (\mu)\, p + f_u(u)\right\},\\
        \HH_c(p) = \sup\limits_{c} \left \{  f_c(c) - c \, p \right \}
    \end{cases}
\end{equation}

\subsection{Numerical Illustration}
        A deterministic, simplified version of the model was simulated through Picard iterations.
        
        \textcolor{red}{describe in more details what you did.}
        
        We aim to use this simulation as a benchmark for further studies.        
        Let's set \textcolor{red}{Parameter table}
        $$g(1- u) = (1 - u),\,\xi = 0,\, \delta = 0.05, \, k = 0.5,\, f_c(c) = \log(c), f_u(u) = -\frac{\alpha}{2} u^2, Q(a) = a.$$        
        In this case, we have 
        \begin{equation}
            \begin{cases}    
            \HH_u(h,p,q) = \sup\limits_{u} \left\{ (1 - u)q + \bar w_t h u p - \frac{\alpha}{2} u^2 \right\},\\
            \HH_c (p) = \sup\limits_{c} \left\{ \log(c) - cp \right\}.
            \end{cases}
        \end{equation} 
        Two scenarios are simulated: $\alpha = 0.5$ and  $\alpha = 2$ representing a lower and a higher preference to education versus work, respectively.

        \textcolor{red}{Try to solve PDEs using DGM}
        \textcolor{red}{Change format of plots to suit document.}
        \includegraphics[width=\textwidth]{simulations_mfg.png}