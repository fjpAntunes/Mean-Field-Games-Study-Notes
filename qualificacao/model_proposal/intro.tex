We propose a game theoretical model for time allocation between work and education.
The model is based on Lucas's human capital growth model~\cite{lucas1988mechanics} and 
Aiyagari’s growth model~\cite{achdou2022income,carmona2018probabilistic}.
The agents interactions are mediated through the interest rate and wage rate of the economy,
which depend on aggregated quantities of the agent's states.
A key difference between the model proposed here and in~\cite{lucas1988mechanics} is heterogeneity of the agents in our model.
The model was motivated by the high rate of school evasion in Brazil due to the necessity of working.
Data from the 2023 Continuous National Household Sample Survey~\cite{pnad2023} show that the necessity of working
is given as reason for not pursuing further education by 45\% of people out of school in between 15 and 29 years old. 
Some of the results expected of this research are qualitative properties
of the population behaviour and numerical solutions of the mean field limit
of the game. Through these results, we expect to be able to model public policies
to promote education.



