This section is a compendium of stochastic control.
We follow~\cite{pham2009continuous} and present briefly the necessary results
for a full understanding of MFG theory.

Consider a system governed by the controlled
stochastic differential equation (SDE) in $\RR^n$:
\begin{equation}\label{soc:controlled_sde}
    d X_s = b(X_s, \alpha_s)ds + \sigma(X_s, \alpha_s) d W_s
\end{equation}
where $W$ is a $d$-dimensional Brownian motion on a filtered probability space
$(\Omega, \mathcal{F}, \mathbb{F} = {(\mathbb{F}_t)}_{t \geq 0}, \mathbb{P} )$ satisfying the
usual conditions.
The coefficients $b(t,x,a)$ and $\sigma(t,x,a)$ are depend on $t$, except
in the case of infinite horizon problems.
The control $\alpha = (\alpha_s)$ is a progressively measurable process w.r.t.
$\mathbb{F}$, valued in $A \subset \RR^m$.

The functions $b: \mathbb{R}^n \times A \mapsto \mathbb{R}^n$ and 
$\sigma: \mathbb{R}^n \times A \mapsto \RR^{n\times d}$ are uniformly
Lipschitz over $A$: $\exists K \geq 0$ such that $\forall x,y, \in \RR^n, \forall a \in A$,
\begin{equation}\label{soc:diffusion_uniformly_lipschitz}
    | b(x,a) - b(y,a) | + | \sigma(x,a) - \sigma(y,a) | \leq K | x - y |.   
\end{equation}

Fix a time $0 < T < \infty$, and let $\mathcal{A}$ be the set of control processes
$\alpha$ such that
\begin{equation}\label{soc:diffusion_quadratic_bound}
    \mathbb{E}\left[ \int^T_0 |b( 0,\alpha_t )|^2 + |\sigma( 0, \alpha_t )|^2 dt \right] < \infty.
\end{equation}

Conditions~\eqref{soc:diffusion_uniformly_lipschitz} 
and~\eqref{soc:diffusion_quadratic_bound} ensure the existence and uniqueness
of a strong solution to the SDE~\eqref{soc:controlled_sde}
for every control $\alpha \in \mathcal{A}$ and for every initial condition
$(t,x) \in [0,T] \times \RR^n$, and we denote by $\{ X_s^{t,x}, t \leq s \leq T \}$
this solution with a.s.\ continuous paths. 

We now turn to the definitions of the objective function and the value function.
Let $ f: [0,T] \times \RR^n \times A \mapsto \RR $ and $ g: \RR^n \mapsto \RR $
be two measurable functions. Assume that $g$ is either upper-bounded or 
satisfies a quadratic growth condition 
$| g(x) | \leq C(1 + |x|^2),\ \forall x \in \RR^n$, for some constant $C$
independent of $x$.

For $(t,x) \in [0,T] \times \RR^n$, let $\mathcal{A}(t,x) \subset \mathcal{A}$
be such that, $\forall \alpha \in \mathcal{A}(t,x)$
\begin{equation}
    \mathbb{E}\left[ \int_t^T | f(s, X_s^{t,x}, \alpha_s) | ds \right] < \infty,
\end{equation}
and assume $\mathcal{A}(t,x)$ is not empty for all
$(t,x) \in [0,T] \times \RR^n$.

We can now define the objective function $J$
\begin{equation}
    J(t, x, \alpha) = \mathbb{E}\left[ \int_t^T f(s, X_s^{t,x}, \alpha_s) ds + g(X^{t,x}_T) \right],
\end{equation}
for all $(t,x) \in [0,T] \times \RR^n$ and $\alpha \in \mathcal{A}(t,x)$.
We want to maximize over the controlled processes the objective function, so
we introduce the associated value function $v$:
\begin{equation}
    v(t,x) = \sup_{\alpha \in \mathcal{A}(t,x)} J(t,x,\alpha).
\end{equation}
Given an initial condition $(t,x) \in \left[ 0,T \right) \times \RR^n$, we say
that $\hat \alpha \in \mathcal{A}(t,x)$ is an optimal control if
$v(t,x) = J(t,x,\hat \alpha)$.
Moreover, a control process $\alpha$ in the form $\alpha_s = a(s,X_s^{t,x})$ for
a measurable function $a: [0,T] \times \RR^n \mapsto A$ is
called a Markovian (or feedback) control.

We are now in position to state a fundamental property of the value function $v$,
called the \textit{Dynamic Programming Principle}:

\begin{theorem}[Dynamic Programming Principle]
    Let $(t,x) \in [0,T] \times \RR^n$. The value function $v$ satisfies
\begin{gather} 
        v(t,x) = \sup_{\alpha \in \mathcal{A}(t,x)}
        \sup_{\theta \in \mathcal{T}_{t,T}}
        \mathbb{E}\left[ \int_t^\theta f(s, X_s^{t,x}, \alpha_s) ds + v(\theta, X^{t,x}_\theta) \right]\\
        =\sup_{\alpha \in \mathcal{A}(t,x)}
        \inf_{\theta \in \mathcal{T}_{t,T}}
        \mathbb{E}\left[ \int_t^\theta f(s, X_s^{t,x}, \alpha_s) ds + v(\theta, X^{t,x}_\theta) \right]
\end{gather}
    where $\mathcal{T}_{t, T}$ is the set of stopping times valued in $[t,T]$.
\end{theorem}

\begin{proof}
    Given an admissible control $\mathcal{A}(t,x)$, by pathwise uniqueness of the flow of the SDE for $X$ we have
    \begin{equation}
        X_s^{t,x} =  X_s^{\theta, X_\theta^{t,x}}, \, s \geq \theta,
    \end{equation}
    for any stopping time $\theta \in [t,T]$. By the tower property of expectations, 
    we have
    \begin{align*}
        J(t,x,\alpha) &= \mathbb{E}\left[ \int_t^T f(s, X_s^{t,x}, \alpha_s) ds + G(X_T^{t,x})  \right] \\
        &= \mathbb{E}\left[ \mathbb{E} \left[ \int_t^T f(s, X_s^{t,x}, \alpha_s) ds + G(X_T^{t,x}) | \mathcal{F}_\theta \right] \right] \\
        &= \mathbb{E}\left[  \int_t^\theta f(s,X_s^{t,x}, \alpha_s) ds +  \mathbb{E} \left[ \int_\theta^T f(s, X_s^{\theta, X_\theta^{t,x}}, \alpha_s) ds + G(X_T^{\theta, X_\theta^{t,x}}) | \mathcal{F}_\theta \right] \right]\\ 
        &= \mathbb{E}\left[  \int_t^\theta f(s,X_s^{t,x}, \alpha_s) ds +  \mathbb{E} \left[ \int_\theta^T f(s, X_s^{t,x}, \alpha_s) ds + G(X_T^{t,x}) | \mathcal{F}_\theta \right] \right]\\ 
        &= \mathbb{E}\left[  \int_t^\theta f(s,X_s^{t,x}, \alpha_s) ds +  J(\theta, X_\theta^{t,x}, \alpha) \right].
    \end{align*}
    By the definition of the value function $v$, $J(t,x,\alpha) \leq v(t,x)$.
    Moreover, as $\theta$ is arbitrary in $\mathcal{T}_{t,T}$, we have
    \begin{align*}
        J(t,x,\alpha) &\leq \inf_{\theta \in \mathcal{T}_{t,T}} \mathbb{E}\left[ \int_t^\theta f(s, X_s^{t,x}, \alpha_s) ds  + v(\theta, X_\theta^{t,x})\right] \\
        & \leq \sup_{\alpha \in \mathcal{A}(t,x)} \inf_{\theta \in \mathcal{T}_{t,T}} \mathbb{E}\left[ \int_t^\theta f(s,X^{t,x}_s, \alpha_s) ds + v(\theta, X_\theta^{t,x}) \right]. 
    \end{align*}
    by taking the supremum over $\alpha$ in the left side, we conclude that
    \begin{equation}
        v(t,x)  \leq \sup_{\alpha \in \mathcal{A}(t,x)} \inf_{\theta \in \mathcal{T}_{t,T}} \mathbb{E}\left[ \int_t^\theta f(s,X^{t,x}_s, \alpha_s) ds + v(\theta, X_\theta^{t,x}) \right]. 
    \end{equation}
    Now, for the other side of the inequality: fix some arbitrary control
    $\alpha \in \mathcal{A}(t,x)$ and $\theta \in \mathcal{T}_{t,T}$.
    By definition of the value functions, for any $\epsilon > 0$ and 
    $\omega \in \Omega$, there exists an $\epsilon$-optimal control
    $\alpha^{\epsilon, \omega} \in \mathcal{A}( \theta(\omega), X^{t,x}_{\theta(\omega)} (\omega)  )$
    such that
\begin{equation}\label{soc:dpp_proof_epsilon_optimal_control}
    v(\theta(\omega), X^{t,x}_{\theta(\omega)}(\omega)) - \epsilon \leq J(\theta(\omega), X^{t,x}_{\theta(\omega)}(\omega), \alpha^{\epsilon, \omega}).
\end{equation}    
    We can define the process
\begin{equation*}
    \hat\alpha_s(\omega) = 
    \begin{cases}
        \alpha_s(\omega), s \in [0,\theta(\omega)],\\
        \alpha_s^{\epsilon, \omega}(\omega), s \in [\theta(\omega), T].
    \end{cases}
\end{equation*}
As stated in~\cite{pham2009continuous}, it can be shown that $\hat \alpha$ is progressively measurable,
so it lies in $\mathcal{A}(t,x)$.
Again, by the tower property and by inequality~\eqref{soc:dpp_proof_epsilon_optimal_control}
\begin{align*}
    v(t,x) &\geq J(t,x, \hat \alpha) = \mathbb{E}\left[ \int_t^\theta f(s,X^{t,x}_s, \alpha_s) ds + J(\theta, X^{t,x}_\theta, \alpha^\epsilon) \right] \\
            &\geq \mathbb{E}\left[ \int_t^\theta f(s,X^{t,x}_s, \alpha_s) ds + v(\theta, X^{t,x}_\theta) \right] - \epsilon.
\end{align*}
As $\alpha, \theta$ and $\epsilon$ were arbitrary, we conclude that
\begin{equation}
    v(t,x) \geq \sup_{\alpha \in \mathcal{A}(t,x)} \sup_{\theta \in \mathcal{T}_{t,T}} \mathbb{E} \left[ \int_t^\theta f(s, X_s^{t,x}, \alpha_s) ds + v(\theta, X^{t,x}_\theta) \right].
\end{equation}
Combining both inequalities concludes the proof.
\end{proof}

From the dynamic programming principle, we can derive the Hamilton-Jacobi-Bellman (HJB)
equation. The HJB equation is an infinitesimal version of the DPP, which describes
the local behaviour of the value function.

Consider the time $\theta = t + h$ and a constant control $\alpha_s = a$, for some
arbitrary $a \in A$. From the statement of the DPP, we have
\begin{equation}\label{soc:dpp_ineq}
    v(t,x) \geq \mathbb{E}\left[ \int_t^{t + h} f(s,X^{t,x}_s,a) ds + v(t+h, X^{t,x}_{t+h}) \right]
\end{equation}
By assuming smoothness of $v$, we can apply Ito's formula between $t$ and $t+ h$:
\begin{equation*}
    v(t+h, X^{t,x}_{t+h}) = v(t,x) + \int^{t+h}_t \left( \frac{\partial v}{\partial t} + \mathcal{L}^a v\right)(s, X^{t,x}_s) ds + \int_t^{t+h} \frac{\partial v}{\partial x} (s, X^{t,x}_s) dW_s,
\end{equation*}
where $\mathcal{L}^a$ is the operator associated with the
diffusion~\eqref{soc:controlled_sde}, defined by
\begin{equation}
    \mathcal{L}^a v = b(x,a) \partial_x v + \frac{1}{2}\text{tr} (\sigma(x,a) \sigma'(x,a) \partial_{xx}v).
\end{equation}
By substituting back into~\eqref{soc:dpp_ineq}, we have
\begin{equation*}
    0 \geq \mathbb{E}\left[ \int^{t+h}_t \left( \frac{\partial v}{\partial t} + \mathcal{L}^a v\right)(s, X^{t,x}_s) + f(s, X^{t,x}_s, a)  ds\right]
\end{equation*}
We can divide by $h$, apply the mean value theorem and take $h\to 0$ to arrive at
\begin{equation*}
    0 \geq \frac{\partial v}{\partial t} (t,x) + \mathcal{L}^a v(t,x) + f(t,x,a).
\end{equation*}
Since this holds for any $a \in A$, we have that
\begin{equation}\label{soc:hjb_inequality}
    - \frac{\partial v}{\partial t} (t,x) - \sup_{a \in A} \left[ \mathcal{L}^a v(t,x) + f(t,x,a) \right] \geq 0.
\end{equation}

On the other hand, suppose that $\alpha^*$ is an optimal control.
Then equality is achieved in the DPP:
\begin{equation}
    v(t,x) = \mathbb{E}\left[ \int^{t+h}_t f(s, X^*_s, \alpha^*_s) ds + v(t + h, X^*_{t+h}) \right],
\end{equation}
where $X^*$ is the evolution of the system starting from $x$ at $t$, with control $\alpha^*$.
By the same procedure as above, we arrive at
\begin{equation}
    - \frac{\partial v}{\partial t} (t,x) - \mathcal{L}^a v(t,x) - f(t,x,a) = 0.
\end{equation}
which, combined with~\eqref{soc:hjb_inequality} allows us to conclude that
\begin{equation}
    - \frac{\partial v}{\partial t} (t,x) - \sup_{a \in A} \left[ \mathcal{L}^a v(t,x) + f(t,x,a) \right] = 0
\end{equation}
for every $(t,x) \in [0,T) \times \RR^n $ where the supremum in $a$ is finite.

If we define the Hamiltonian function $H : [0,T] \times \RR^n \times \RR^n \times \mathcal{S}_n \mapsto \RR$
by
\begin{equation}
    H(t,x,p,M) = \sup_{a \in A} \left[ b(x,a)\cdot p + \frac{1}{2} \text{tr} (\sigma \sigma' (x,a) M) + f(t,x,a) \right],
\end{equation}
we can rewrite the PDE as
\begin{equation}\label{soc:hjb_equation}
    -\frac{ \partial v}{\partial t} (t,x) - H(t,x,\partial_x v(t,x), \partial_{xx} v(t,x)) = 0
\end{equation}
which is the Hamilton-Jacobi-Bellman equation.
The PDE has a terminal condition given by
\begin{equation}
    v(T,x) = g(x), \forall x \in \RR^n,
\end{equation}
which results from the definition of $v$ at the time horizon $T$.

In order to derive the Hamilton-Jacobi-Bellman equation, we assumed smoothness
of the value function. However, this does not necessarily hold, even in simple
cases. To deal with this difficulty, the notion of viscosity solutions was 
introduced, which allows a rigorous formulation for the HJB to be given under
the assumption of local boundedness. Moreover, viscosity solutions satisfy a 
comparison principle, from which we can derive uniqueness of solutions.
We shall state the definition and uniqueness result here, and
 refer to~\cite{pham2009continuous,bressan2007introduction,lopes1997introduccao} for a deeper exposition, 
 and to~\cite{crandall1992user} for a seminal reference to the topic.

 Let's consider the first order case, with a PDE of the form
 \begin{equation*}
    F(x,u, D u) = 0, x \in \Omega \subset \RR^n
 \end{equation*}
the theory of viscosity solutions show that the solutions $u^\epsilon$
to the equation
\begin{equation*}
    F(x, u^\epsilon, D u^\epsilon) = \epsilon \Delta u^\epsilon
\end{equation*}
converge to a unique limit $u$ as $\epsilon \to 0^+$.
This limit is the viscosity solution to the PDE.
Moreover, the viscosity solution $u$ can be uniquely characterized by inequalities
in the superdifferential set $D^+ u(x)$ and the subdifferential set $D^- u(x)$
as follows:
\begin{itemize}
    \item A function $u \in C(\Omega)$ is a viscosity subsolution if
\begin{equation*}
    F(x, u(x), p) \leq 0, \text{ for every } x \in \Omega, p \in D^+ u(x).
\end{equation*}
    \item A function $u \in C(\Omega)$ is a viscosity supersolution if
\begin{equation*}
    F(x, u(x), p) \geq 0, \text{ for every } x \in \Omega, p \in D^- u(x).
\end{equation*}
    \item A function $u \in C(\Omega)$ is a viscosity solution if it is both
    a viscosity supersolution and a viscosity subsolution.
\end{itemize}
We can characterize the superdifferential (respectively the subdifferential) in
terms of test functions $\phi \in C^1(\Omega)$:
\begin{definition}
    Let $u \in C(\Omega)$ and $x_0 \in \Omega$. Then $p_0  \in D^+ u(x_0)$ if and 
    only if there exists $\phi \in C^1(\Omega)$ such that:
\begin{enumerate}
    \item $u - \phi$ has a local maximum in $x_0$;
    \item $(u - \phi)(x_0) = 0$;
    \item $D \phi(x_0) = p_0$;
\end{enumerate}
\end{definition}
For $p_0 \in D^- u(x_0)$, instead of a local maximum we have a local minimum.
This allows us to give an equivalent definition of viscosity solutions in terms 
of test functions. In fact, for second-order PDEs
of the form
\begin{equation}
    F(x, u, D u, D^2 u) = 0,
\end{equation}
where $F$ is assumed to satisfy the ellipticity condition
\begin{equation}
    M \leq \hat M \rightarrow F(x,r,p,M) \geq F(x,r,p,\hat M),
\end{equation}
for all $x \in \Omega, r \in \RR, p \in \RR^n, M,\hat M \in \mathcal{S}_n$,
this definition can be extended by considering test functions in $C^2(\Omega)$:
\begin{definition}[Second Order Viscosity Solutions]
    \begin{itemize}
        \item An upper-semicontinuous function $u : \Omega \mapsto \RR$ is a viscosity subsolution if
    \begin{equation*}
        F(x_0, u(x_0), D \phi(x_0), D^2 \phi(x_0)) \leq 0,
    \end{equation*}
    for all $x_0 \in \Omega$ and for all $\phi \in C^2(\Omega)$ such that $x_0$ is a maximum point of $u - \phi$.
        \item A lower-semicontinuous function $u : \Omega \mapsto \RR$ is a viscosity supersolution if
    \begin{equation*}
        F(x_0, u(x_0), D\phi(x_0), D^2 \phi(x_0)) \geq 0,
    \end{equation*}
    for all $x_0 \in \Omega$ and for all $\phi \in C^2(\Omega)$ such that $x_0$ is a minimum point of $u - \phi$.
        \item A function $u : \Omega \mapsto \RR$ is a viscosity solution if it is both
        a viscosity supersolution and a viscosity subsolution.
    \end{itemize}
\end{definition}