

Mean field games are a framework for approximating Nash equilibria of certain kinds of games with a large number of players.
There are two equivalent ways of formulating a mean field game - through a system of coupled partial differential equations, or through a system of forward-backward stochastic differential equations.

In this section, we will explore the theory of mean field games.
We will begin by reviewing important prerequisites - namely stochastic optimal control in . 
Then, we shall review both the PDE system point of view and the FBSDE point of view. 
Equiped with an understanding of both point of views, we will analyze the limit of N-player games and its relationship to MFG.
Following the theoretical subsections, we will explore numerical solutions for MFG and applications.
We shall end the section by analyzing a model for human capital development.
We refer the reader to Appendix \ref{appendix-game-theory} for a review of game theory concepts, and to Appendix \ref{appendix-mathematical-background} for a review of necessary mathematical background.


\if{
\subsection{Three roads for Mean Field Games Theory}

Conceptually, there are three approaches to mean field games theory - from physics, from game theory and from economic theory.

\subsubsection{First road: from physics to mean field games}

% To do
% Add an physical example - i.e. gravity vs termal pressure model
In particle physics, situations with large number of particles are handled using mean field theory. Instead of modelling all the inter-particle interactions, one introduces a "mean field" which serve as mediator for the interactions. Each particle both contributes to and is influenced by the mean field.

In order to use this approximation, the inter-particle interactions must be sufficiently weak or regular.

Mean field game theory adapts this methodology to situations in which agents interact in strategic situations. The challenge is to take into account not only the ability of agents to make decisions, but also the interaction between strategies: each player's strategy tries to take into account the other player's strategy.
\textbf{This changes the nature of the mean field: it is not an statistic on the domain of particle states anymore, but rather a statistic on the domain of strategies and information.}

Although the term "mean-field" is borrowed from Physics, mean field game theory does not restrain itself to applying physical models to economy. Instead, as a branch of game theory, mean field games models aim to \textit{explain} rational behaviour through the structure of the agent's interactions and payoffs. \cite{cousin2010paris}

\subsubsection{Second Road: From Game Theory to Mean Field Games}

In game theory, $N$-player games quickly become intractable as $N$ gets large. Mean Field Games provide a way to approximate the limiting case as $N \to \infty$ for a class of $N$-player game which respect a form of anonymity: direct interactions between players are comparatively small, and players can be interchanged without changing the interaction. This is the case when interactions are mediated through some average of the player's state, for instance.
The mean field approach consists in approximating the $N$ players by a continuum of players distributed through the state space. Each agent formulates his optimal response given the distribution of players, and conversely this optimal response implies an evolution through time for the player's distribution.
\textit{In layman's terms, each player formulates his strategy against the crowd, and the crowd evolves according to each player's strategy.}

\subsubsection{Third Road: From Economics to Mean Field Games}
% To Do
% Add toy model with general equilibrium, reference it here
In economic models following the Theory of General Economic Equilibrium, interactions are mediated by prices. Direct interactions of agents are excluded from these economic models. However, systemic economic effects such as externalities, public goods, etc give rise to interactions which are not mediated by price. Both price formation and systemic effects can be modelled by a mean field type model. This way, MFG theory lends itself as a tool of economic analysis.
}\fi