PLACEHOLDER
References : Cardaliaguet Notes, Ambrosio book, 
Wasserstein distance and Topology in Probability Measure Spaces

Let $\Omega \subset \RR^n$, $\mathcal{P}(\Omega)$ be the set of Borel
probability measures on $X$, and $\mathcal{P}_2(\Omega) \subset \mathcal{P}(\Omega)$
be the set of probability measures $m$ such that $\int_\Omega |x|^2 dm(x) < + \infty$.

We can define a metric over $\mathcal{P}_2(\Omega)$ called the 
\textit{Wasserstein distance} by
\begin{equation}\label{prob_measures:wasserstein_distance}
    d_2(\mu, \nu) = \inf_{\gamma \in \Pi(\mu,\nu)} \left[ \int_\Omega |x - y|^2 d\gamma(x,y) \right]^{\frac{1}{2}}
\end{equation}
where $\Pi(\mu,\nu)$ is the \textit{coupling} between $\mu$  and $\nu$,
that is, the set of Borel probability measures on $\Omega \times \Omega$
such that $\gamma(A \times \Omega) = \mu(A)$ and $\gamma(\Omega \times A) = \nu(A)$
for any Borel set $A \subset \Omega$.

The Wassertestein distance is a special case of Monge-Kantorovich distances
\begin{equation*}
    d_p(\mu, \nu) = \inf_{\gamma \in \Pi(\mu, \nu)} \left[ \int_{\Omega^2} d^p(x,y) d\gamma(x,y)  \right]^{\frac{1}{p}}.
\end{equation*}
One important property of Monge-Kantorovich distances is the existence of at least
one optimal measure, that is, one $\hat  \gamma \in \Pi(\mu,\nu)$ such that
$\d_p(m,m') = \left[ \int_{\Omega^2} d^p(x,y) d {\hat \gamma}(x,y) \right]^{\frac{1}{p}}$.
The measure $\hat \gamma$ is called optimal transport plan.

In the case of the Wasserstein distance, an stronger property is satisfied when 
$\mu$ is absolutely continuous: the optimal transport plan can be realized as
an \textit{optimal transport map}
\begin{theorem}[Existence of an optimal transport map]
    If $\mu \in \mathcal{P}_2(\Omega)$ is absolutely continuous, then,
    for any $\nu \in \mathcal{P}_2(\Omega)$, there exists a convex map 
    $\Phi: \Omega \mapsto \RR $ such that the measure
    $ (\text{id}, D\Phi)\sharp \mu $ is optimal for $d_2(\mu, \nu)$, and $\nu = D \Phi \sharp \mu$.

    Conversely, if the convex map $\Phi : \Omega \mapsto \RR$ satisfies $\nu = D \Phi \sharp \mu$,
    then the measure $(\text{id}, D\Phi)\sharp \mu$ is optimal for $d_ 2(\mu,\nu)$.
\end{theorem}
where the symbol $\sharp$ denotes the push-forward of the measure with respect to the map,
that is, for $\psi : \Omega \mapsto \Omega$, $\psi \sharp \mu(A) = \mu(\psi^{-1}(A))$


Flows in Wasserstein Space

\begin{definition}[Absolutely Continuous Curves in Wasserstein space]
    Let $v : (a,b) \mapsto \mathcal{P}_2(\Omega)$ be a curve.
    We say that $v$ is absolutely continuous if there exists
    $m \in L^1(a,b)$ such that
    \begin{equation}\label{wass:absolute_continuity}
        d(v(s), v(t)) \leq \int_s^t m(r) dr, \forall a < s \leq t < b.
    \end{equation}
\end{definition}

Among all choices of $m$ in~\eqref{wass:absolute_continuity}, there is a minimal
one, which describes the \textit{metric derivative} of the curve
\begin{theorem}[Metric derivative]
    For any absolute continuous curve $v$ over $\mathcal{P}_2$, the limit
    \begin{equation}
        |v'|(t) = \coloneqq \lim_{s \to t} \frac{d(v(s), v(t))}{|s - t|}
    \end{equation}
    exists for $\mathcal{L}$-a.e. $t \in (a,b)$. Moreover, the function
    $ t \mapsto |v'|(t) $ belongs to $L^1(a,b)$, is an admissible integrand for
    the right hand side of~\eqref{wass:absolute_continuity}, and it satisfies
    \begin{equation}
        |v'|(t) \leq m(t) \text{ for }\mathcal{L}\text{-a.e. } t \in (a,b),
    \end{equation}
    for each function $m$ satisfying~\eqref{wass:absolute_continuity}.
\end{theorem}

\begin{theorem}[Absolutely continuous curves and the continuity equation]
    Let $\mu_t : (a,b) \mapsto \mathcal{P}_2(\Omega)$ be an absolutely continuous
    curve and let $|\mu'| \in L^1(a,b)$ be its metric derivative.
    Then there exists a Borel vector field $v : (x,t) \mapsto v_t(x)$ such that
    \begin{equation}
        v_t \in L^2(\mu_t, \Omega), ||v_t||_{L^2(\mu_t, \Omega)} \leq |\mu'|(t)\text{ for a.e. } t \in (a,b)
    \end{equation}
    and $\mu_t$ is a weak solution to the continuity equation
    \begin{equation}
        \partial_t \mu_t + \nabla \cdot (v_t \mu_t) = 0.
    \end{equation}
    Conversely, if a narrowly continuous curve 
    $\mu_t : (a,b) \mapsto \mathcal{P}_2(\Omega)$
    satisfies the continuity equation for some Borel velocity field $v_t$ with
    $||v_t||_{L^1(\mu_t, \Omega)} \in L^1(a,b)$ then $mu_t$ is absolutely continuous
    and $ |\mu'|(t) \leq  ||v_t||_{L^1(\mu_t, \Omega)}$ for a.e. $t \in (a,b)$.
\end{theorem}

This theorem shows that the trajectories of absolute continuous curves are described
by the continuity equation. In fact, the Wasserstein distance is also characterized
by solutions of the continuity equation through the Benamou-Brenier formula.

\begin{theorem}[Benamou-Brenier Formula]
    For all  $\mu_0, \mu_1 \in \mathcal{P}_2*(\Omega)$, we have
    \begin{equation}
        d_2(\mu_0, \mu_1) = \min \left\{ \int_0^1 ||v_t||^2_{L^2(\mu_t, \Omega)} \, dt : \partial_t \mu_t + \div (v_t \mu_t) = 0 \text{ in } (0,1) \times \Omega \right\}
    \end{equation} 
    where the minimization is among all curves $\mu_t : [0,1] \mapsto \mathcal{P}_2(\Omega)$
    continuous w.r.t. the weak topology.
\end{theorem}

In the theory of mean-field games, it is convenient to define some
form of differentiability for functions
 $\mathcal{U} : \mathcal{P}_2(\Omega) \mapsto \RR$ of probability measures.
 This is not straightforward to do, as the Wasserstein space lacks a vector space
 structure.
There are at least three aproaches to define derivatives $D_m U$ of functions in
Wassertein spaces: 
Ambrosio (Add citation) defines a kind of manifold structure
on the Wasserstein space;
Cardaliaget (Add citation) restricts the 
function $U$ to elements of $\mathcal{P}_2(\Omega)$ which have a density in 
$L^2(\Omega)$ to use its Hilbert structure; and 
Lyons (add citation) introduces a "lifted" function $\tilde u$ over the space of $L^2$ random variables with values
in $\Omega$ such that $\tilde u(X) = U(\mathcal{L}(X))$, and uses the Hilbert structure
of the space of $L^2$ random variables.

However, the important property of the derivative $D_m U$ the following:
Consider an absolutely continuous measure flow in Wasserstein space $\mu_t$
described by $\partial_t \mu_t + \div(v_t \mu_t) = 0$,
where $v_t$ is a vector field.
The map $ t \mapsto U(\mu_t)$ is a map from $\RR$ to $\RR$,
and if $D_m U$ exists it follows that
\begin{equation}\label{wass:measure_derivative_chain_rule}
    \frac{d}{dt} U(\mu_t) = \int_\Omega D_m U(\mu_t, y) \cdot v_t(y) d\, \mu_t(y)
\end{equation}

\begin{proposition}
    Given a $C^1$ function $U : \mathcal{P}_2(\Omega) \mapsto \RR$,
    if $D_m U$ exists, it is unique.
\end{proposition}
\begin{proof}
    Suppose that there are two operators 
    $A, B: \mathcal{P}_2(\Omega) \times \Omega \mapsto \RR^d$
    satisfying the property~\eqref{wass:measure_derivative_chain_rule}.
    Fix a measure $\mu$, and for a vector field 
    $v_t \in \RR^d$, $t \in (-\epsilon, \epsilon)$
    let $\mu_t$ be the solution to $\partial_t \mu_t + \div(v_t \mu_t) = 0$
    with $\mu_0 = \mu$.
    Now, from~\eqref{wass:measure_derivative_chain_rule}, it follows that
    \begin{align*}
    \frac{d}{dt} U(\mu_t) |_{t = 0} = \int_\Omega A(\mu, y) \cdot v_0(y) d\, \mu(y) = \int_\Omega B(\mu, y) \cdot v_0(y) d\, \mu(y)\\
    \rightarrow \int_\Omega [A(\mu, y) - B(\mu, y)]  \cdot v_0(y) d\, \mu(y) = 0
    \end{align*}
    Note to self: maybe not so simple - what if $\int_\Omega [A(\mu, y) - B(\mu, y)]  \cdot v_0(y) d\, \nu(y) \neq 0$?

\end{proof}


Derivation of MFG system from $N$-player game (following Cardaliaget)
Consider a differential game with $N$ players in $\RR^d$ where each player
controls his velocity. In this setting, the state of player $i$
 evolves according to
\begin{equation}
    x'_i(t) = \alpha_i(t).
\end{equation}
The cost functional for player $i$ is of the form
\begin{equation}
    J_i(x, t, (\alpha_j)_j) = \int_t^T L_i (x_1(s), \dots, x_N(s), \alpha_i(s)) ds + g_i(x_1(T), \dots, x_N(T)).
\end{equation}

We assume that
\begin{equation}
    L_i(x_1, \dots, x_N, \alpha) = \frac{1}{2}|\alpha|^2 + F\left( \frac{1}{N-1} \sum_{j \neq i} \delta_{x_j}  \right)
\end{equation}
where $F : \mathcal{P}_2 \mapsto \RR$ is continuous, and
\begin{equation}
    g_i(x_1, \dots, x_N) = g(x_i, \frac{1}{N-1} \sum_{j\neq i} \delta_{x_j})
\end{equation}
where $g: \RR^d \times \mathcal{P}_2$ is continuous.

We assume that a smooth, symmetric Nash equilibrium in feedback form exists for
this game - that is, there is a map
$U^N : \RR^d \times [0,T] \times (\RR^d)^{(N-1)} \mapsto \RR$ such that
\begin{equation}
    U^N_i(x_i, t, (x_j)_{j\neq i}) = U^N(x_i, t, (x_j)_{j\neq i})
\end{equation}
satisfies the system of $HJ$ equations:
\begin{equation}
    -\frac{\partial U_i^N}{\partial t} + \frac{1}{2}|D_{x_i} U_i^N|^2 - F\left( \frac{1}{N-1}  \sum_{j\neq i} \delta_{x_j} \right) + \sum_{j \neq i} \langle D_{x_j} U^N_j, D_{x_j} U^N_i \rangle = 0
\end{equation}
as stated in \cite{Cardaliaget}, the family of feedbacks $(\alpha_i(x,t) = - D_{x_i} U^N_i (x,t))$
provides a Nash equilibrium for the game.

Now, assume that $U^N$ satisfy the following estimates
\begin{equation}
    \sup_{x_1, t, (x_j)_j \leq 2} \left| D_{x_1, t} U^N (x_1, t, (x_j)) \right| \leq C,
\end{equation}
and
\begin{equation}
    \sup_{x_1, t, (x_j)_j \leq 2} \left| D_{x_j} U^N (x_1, t, (x_j)) \right| \leq  \frac{C}{N}, \text{ for } j\neq 2.
\end{equation}
Under these conditions, and up to a subsequence, there is a map 
$U : \RR^d \times [0,T] \times \mathcal{P}_2 \mapsto \RR$ such that,
for any $R > 0$,
\begin{equation}
    \sup_{|x| \leq R, t, (x_j)_{j\geq 2}} | U^N(x,t,m^{N-1}_{(x_j)}) - U(x,t,m^N_x) | \to 0
\end{equation}
where as before, we have set
\begin{equation}
    m^{N-1}_{(x_j)} = \frac{1}{N-1} \sum_{j \geq 2} \delta_{x_j}, \text{ and } m^N_x = \frac{1}{N} \sum_{j = 1}^N \delta_{x_j}.
\end{equation}
As stated in \cite{Cardaliaget}, we have
\begin{equation}
    \frac{\partial U^N_i}{\partial t} \to \frac{\partial U(x,t,m)}{\partial t}, \quad |D_{x_i} U_i^N|^2 \to |D_x U (x,t,m)|^2.
\end{equation}
It can also be proved \cite{Cardaliaget reference 53} that
\begin{equation}
    \sum_{j\neq i} \langle D_{x_j} U^N_j , D_{x_j} U^N_i \rangle \to \langle D_m U(\cdot,t,m), D_x U(\cdot, t, m) \rangle_{L^2_m}.
\end{equation}
So far we have heuristically explained that the limit of $U^N$ as $N \to \infty$
is some $U\in \mathcal{C}^0(\RR^d \times [0,T] \times \mathcal{P}_2) $
 which satisfies 
 \begin{equation}\label{wass:simple_master_equation}
    \begin{cases}
        - \frac{\partial U}{ \partial t} + \frac{1}{2} |D_x U(x,t,m)|^2 - F + \langle D_m U, D_x U \rangle_{L^2_m} = 0\text{ in }\RR^d \times [0,T] \times \mathcal{P}_2,\\
        U(x,T,m) = g(x,m)
    \end{cases}
 \end{equation}
 Note that this differential equation includes a differential with respect to^
 a probability measure.
  Let's apply an idea similar to the method of characteristics to this equation:
  let $m(t)$ be an absolute continuous measure flow over $\mathcal{P}_2$, and
  let $u(x,t) = U(x,t,m(t))$. We have that
\begin{equation}
    \frac{\partial u}{\partial t} = \frac{\partial U}{\partial t} + \langle D_m U (\cdot, t, m(t)), v(t) \rangle_{L^2_{m(t)}},
\end{equation}
    where $v(t)$ is the vector field driving $m(t)$ through the continuity equation $\patial_t m + \div( m(x,t) v(x,t) ) = 0$
    We can rewrite equation~\eqref{wass:simple_master_equation} as 
\begin{equation}
    \frac{\partial U}{ \partial t} + \langle D_m U, - D_x U \rangle_{L^2_m} = \frac{1}{2} |D_x U(x,t,m)|^2 - F.
\end{equation}
    Therefore, if we \textit{choose} as driver of $m(t)$ the vector field $v(x,t) = - D_x U(x,t,m(t))$,
    we have
\begin{equation}
    \frac{\partial u}{\partial t} =  \frac{\partial U}{ \partial t}(x,t,m(t)) + \langle D_m U(\cdot, t, m(t) ), - D_x U(\cdot, t, m(t) ) \rangle_{L^2_m} = \frac{1}{2} |D_x u(x,t)|^2 - F
\end{equation}
    from which we arrive at the MFG system of PDEs:
\begin{equation}
    \begin{cases}
        - \frac{\partial u}{\partial t} + \frac{1}{2} |D_x u(x,t)|^2  = F(m), \\
        \partial_t m - \div( \partial_x u(x,t) m(x,t) ) = 0,\\
        m(0) = m_0, \, u(x,T) = g(x, m(T))
    \end{cases}
\end{equation}
    where the first equation holds in the viscosity sense, and the second
    one holds in the sense of distributions.
    The vector field $v(x,t) = -D_x U(x,t,m(t))$ is aptly called the \textit{decoupling field}.