This section is a compendium of the topology and differential structure of
Wasserstein spaces. We follow~\cite{cardaliaguet2010notes,ambrosio2005gradient,ambrosio2021lectures}.
Our goal is to characterize absolute continuous measure flows as solutions to the continuity equation,
and to define the derivative of function in Wasserstein space with respect to the measure.

Let $\Omega \subset \RR^n$, $\mathcal{P}(\Omega)$ be the set of Borel
probability measures on $X$, and $\mathcal{P}_2(\Omega) \subset \mathcal{P}(\Omega)$
be the set of probability measures $m$ such that $\int_\Omega |x|^2 dm(x) < + \infty$.

We can define a metric over $\mathcal{P}_2(\Omega)$ called the 
\textit{Wasserstein distance} by
\begin{equation}\label{prob_measures:wasserstein_distance}
    d_2(\mu, \nu) = \inf_{\gamma \in \Pi(\mu,\nu)} \left[ \int_\Omega |x - y|^2 d\gamma(x,y) \right]^{\frac{1}{2}}
\end{equation}
where $\Pi(\mu,\nu)$ is the \textit{coupling} between $\mu$  and $\nu$,
that is, the set of Borel probability measures on $\Omega \times \Omega$
such that $\gamma(A \times \Omega) = \mu(A)$ and $\gamma(\Omega \times A) = \nu(A)$
for any Borel set $A \subset \Omega$.

The Wassertestein distance is a special case of Monge-Kantorovich distances
\begin{equation*}
    d_p(\mu, \nu) = \inf_{\gamma \in \Pi(\mu, \nu)} \left[ \int_{\Omega^2} d^p(x,y) d\gamma(x,y)  \right]^{\frac{1}{p}}.
\end{equation*}
One important property of Monge-Kantorovich distances is the existence of at least
one optimal measure, that is, one $\hat  \gamma \in \Pi(\mu,\nu)$ such that
$d_p(m,m') = \left[ \int_{\Omega^2} d^p(x,y) d {\hat \gamma}(x,y) \right]^{\frac{1}{p}}$.
The measure $\hat \gamma$ is called optimal transport plan.

In the case of the Wasserstein distance, an stronger property is satisfied when 
$\mu$ is absolutely continuous: the optimal transport plan can be realized as
an \textit{optimal transport map}~\cite{cardaliaguet2010notes}.
\begin{theorem}[Existence of an optimal transport map]
    If $\mu \in \mathcal{P}_2(\Omega)$ is absolutely continuous, then,
    for any $\nu \in \mathcal{P}_2(\Omega)$, there exists a convex map 
    $\Phi: \Omega \mapsto \RR $ such that the measure
    $ (\text{id}, D\Phi)\sharp \mu $ is optimal for $d_2(\mu, \nu)$, and $\nu = D \Phi \sharp \mu$.

    Conversely, if the convex map $\Phi : \Omega \mapsto \RR$ satisfies $\nu = D \Phi \sharp \mu$,
    then the measure $(\text{id}, D\Phi)\sharp \mu$ is optimal for $d_ 2(\mu,\nu)$.
\end{theorem}
where the symbol $\sharp$ denotes the push-forward of the measure with respect to the map,
that is, for $\psi : \Omega \mapsto \Omega$, $\psi \sharp \mu(A) = \mu(\psi^{-1}(A))$

Definition~\eqref{wass:absolutely_continuous_curves} 
and theorems~(\ref{wass:metric_derivative},~\ref{wass:abs_continuity_and_continuity_eq}) are found in~\cite{ambrosio2005gradient}. 

\begin{definition}[Absolutely Continuous Curves in Wasserstein space]\label{wass:absolutely_continuous_curves}
    Let $v : (a,b) \mapsto \mathcal{P}_2(\Omega)$ be a curve.
    We say that $v$ is absolutely continuous if there exists
    $m \in L^1(a,b)$ such that
    \begin{equation}\label{wass:absolute_continuity}
        d(v(s), v(t)) \leq \int_s^t m(r) dr, \forall a < s \leq t < b.
    \end{equation}
\end{definition}

Among all choices of $m$ in~\eqref{wass:absolute_continuity}, there is a minimal
one, which describes the \textit{metric derivative} of the curve
\begin{theorem}[Metric derivative]\label{wass:metric_derivative}
    For any absolute continuous curve $v$ over $\mathcal{P}_2$, the limit
    \begin{equation}
        |v'|(t) \coloneqq \lim_{s \to t} \frac{d(v(s), v(t))}{|s - t|}
    \end{equation}
    exists for $\mathcal{L}$-a.e. $t \in (a,b)$. Moreover, the function
    $ t \mapsto |v'|(t) $ belongs to $L^1(a,b)$, is an admissible integrand for
    the right hand side of~\eqref{wass:absolute_continuity}, and it satisfies
    \begin{equation}
        |v'|(t) \leq m(t) \text{ for }\mathcal{L}\text{-a.e. } t \in (a,b),
    \end{equation}
    for each function $m$ satisfying~\eqref{wass:absolute_continuity}.
\end{theorem}

\begin{theorem}[Absolutely continuous curves and the continuity equation]\label{wass:abs_continuity_and_continuity_eq}
    Let $\mu_t : (a,b) \mapsto \mathcal{P}_2(\Omega)$ be an absolutely continuous
    curve and let $|\mu'| \in L^1(a,b)$ be its metric derivative.
    Then there exists a Borel vector field $v : (x,t) \mapsto v_t(x)$ such that
    \begin{equation}
        v_t \in L^2(\mu_t, \Omega), ||v_t||_{L^2(\mu_t, \Omega)} \leq |\mu'|(t)\text{ for a.e. } t \in (a,b)
    \end{equation}
    and $\mu_t$ is a weak solution to the continuity equation
    \begin{equation}
        \partial_t \mu_t + \nabla \cdot (v_t \mu_t) = 0.
    \end{equation}
    Conversely, if a narrowly continuous curve 
    $\mu_t : (a,b) \mapsto \mathcal{P}_2(\Omega)$
    satisfies the continuity equation for some Borel velocity field $v_t$ with
    $||v_t||_{L^1(\mu_t, \Omega)} \in L^1(a,b)$ then $mu_t$ is absolutely continuous
    and $ |\mu'|(t) \leq  ||v_t||_{L^1(\mu_t, \Omega)}$ for a.e. $t \in (a,b)$.
\end{theorem}

Theorem~\eqref{wass:abs_continuity_and_continuity_eq} shows that the trajectories of absolute continuous curves are described
by the solutions in the sense of distributions to the continuity equation. In fact, the Wasserstein distance is also characterized
by solutions of the continuity equation through the Benamou-Brenier formula.

\begin{theorem}[Benamou-Brenier Formula]
    For all  $\mu_0, \mu_1 \in \mathcal{P}_2*(\Omega)$, we have
    \begin{equation}
        d_2(\mu_0, \mu_1) = \min \left\{ \int_0^1 ||v_t||^2_{L^2(\mu_t, \Omega)} \, dt : \partial_t \mu_t + \nabla \cdot (v_t \mu_t) = 0 \text{ in } (0,1) \times \Omega \right\}
    \end{equation} 
    where the minimization is among all curves $\mu_t : [0,1] \mapsto \mathcal{P}_2(\Omega)$
    continuous w.r.t. the weak topology.
\end{theorem}

Another interesting property of weak solutions of the continuity equation is their relation to solutions $x_t$ of ODEs such as
\begin{equation}
    \dot x = b(x,t).
\end{equation}
If we define the flow 
\begin{equation}
    \Phi(x,t,s) = x + \int_t^s b(x_\tau, \tau) d\tau, 
\end{equation}
then the measure $\mu_s = \Phi(\cdot, 0, s) \sharp \mu_0$ is the unique weak solution to the boundary value problem, as stated in~\cite{cardaliaguet2010notes}:
\begin{equation}
    \begin{cases}
        \partial_t \mu(x,t) - \nabla \cdot ( b(x,t) \mu(x,t) ) = 0,\\
        \mu(x,0) =  \mu_0(x). 
    \end{cases}
\end{equation}


In the theory of mean-field games, it is convenient to define some
form of differentiability for functions
 $\mathcal{U} : \mathcal{P}_2(\Omega) \mapsto \RR$ of probability measures.
 This is not straightforward to do, as the Wasserstein space lacks a vector space
 structure.
There are at least three aproaches to define derivatives $D_m U$ of functions in
Wassertein spaces: 
Ambrosio~\cite{ambrosio2005gradient} defines a kind of manifold structure
on the Wasserstein space;
Cardaliaget~\cite{cardaliaguet2019master} restricts the 
function $U$ to elements of $\mathcal{P}_2(\Omega)$ which have a density in 
$L^2(\Omega)$ to use its Hilbert structure; and 
Lyons~\cite{cardaliaguet2010notes} introduces a "lifted" function $\tilde u$ over the space of $L^2$ random variables with values
in $\Omega$ such that $\tilde u(X) = U(\mathcal{L}(X))$, and uses the Hilbert structure
of the space of $L^2$ random variables.

One property of the derivative $D_m U$ is that it satisfies a form of chain rule with respect to absolute continuous flows of measures. We adopt this property as definition, for it satisfies our needs.
\begin{definition}
We say that a function $U : \mathcal{P}_2(\Omega) \mapsto \RR$ is differentiable at $\mu$ if there exists an element $D_m(\mu, \cdot) \in L^2_\mu$ such that for every solution $\mu_t$ to $\partial_t \mu_t + \nabla \cdot(v_t \mu_t) = 0, \mu_0 = \mu$ it follows that
\begin{equation}\label{wass:measure_derivative_chain_rule}
    \frac{d}{dt} U(\mu_t)_{t = 0} = \int_\Omega D_m U(\mu, y) \cdot v_0(y) d\, \mu(y)
\end{equation}
We say that $U$ is $C^1$ in a neighborhood of $\mu$ if there is a neighborhood of $\mu$ where it is differentiable.
\end{definition}

\begin{proposition}
    Let $U : \mathcal{P}_2(\Omega) \mapsto \RR$,
    be $C^1$ in a neighborhood $N(\mu_0)$ of $\mu$.
    Then, for every $\mu \in N(\mu_0)$, $D_m(\mu,\cdot)$ is uniquely defined
    as an element of $L^2_\mu$.
\end{proposition}
\begin{proof}
    Suppose that there are two operators 
    $A, B: \mathcal{P}_2(\Omega) \times \Omega \mapsto \RR^d$
    satisfying the property~\eqref{wass:measure_derivative_chain_rule}.
    Fix a measure $\mu$, and for a vector field 
    $v_t \in \RR^d$, $t \in (-\epsilon, \epsilon)$
    let $\mu_t$ be the solution to $\partial_t \mu_t + \nabla \cdot(v_t \mu_t) = 0$
    with $\mu_0 = \mu$.
    Now, from~\eqref{wass:measure_derivative_chain_rule}, it follows that
    \begin{align*}
    \frac{d}{dt} U(\mu_t) |_{t = 0} = \int_\Omega A(\mu, y) \cdot v_0(y) d\, \mu(y) = \int_\Omega B(\mu, y) \cdot v_0(y) d\, \mu(y)\\
    \rightarrow \int_\Omega [A(\mu, y) - B(\mu, y)]  \cdot v_0(y) d\, \mu(y) = 0
    \end{align*}
    As $v_0$ was arbitrary, $||A - B||_{L^2_\mu} = 0$, from which we conclude 
    uniqueness.
\end{proof}
