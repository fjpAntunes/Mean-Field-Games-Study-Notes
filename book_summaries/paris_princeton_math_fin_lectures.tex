\section{Introduction}
MFG is a branch of game theory -> Set of concepts, mathematical tools, theorems, simulations methods and algorithms intended to model situations where agents make decisions strategically.

Applied in economy, sociology, engineering, architecture or urban planning.

"Toy models" -> game theoretical models are not supposed to be taken literally, or applied directly to real life situation. In some sense, they're more like fables. Take Prisoner's dilemma, for instance: shows an archetype of strategic interactions that can be recognized in many different situations.
Introducing MFG through toy models is a didatic way of presenting the concepts and mathematics of the theory.
\subsection{All the roads lead to mean field games}

\subsubsection{First road: from physics to mean field games}
In particle physics, situations with large number of particles are handled using mean field theory. Instead of modelling all the inter-particle interactions, one introduces a "mean field" which serve as mediator for the interactions. Each particle both contributes to and is influenced by the mean field.

In order to use this approximation, the inter-particle interactions must be sufficiently weak or regular.

Mean field game theory adapts this methodology to situations in which agents interact in strategic situations. The challenge is to take into account not only the ability of agents to make decisions, but also the interaction between strategies: each player's strategy tries to take into account the other player's strategy.
\textbf{This changes the nature of the mean field: it is not an statistic on the domain of particle states anymore, but rather a statistic on the domain of strategies and information.}

This route sheds more light into the methodology, i