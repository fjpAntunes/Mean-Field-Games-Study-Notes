\documentclass{article}
\usepackage[utf8]{inputenc}
\usepackage{amsmath}
\usepackage{amssymb}
\usepackage{amsthm}
\usepackage{amscd}
\usepackage{amsfonts}
\usepackage{mathtools}
\usepackage{amsmath}
\usepackage{amssymb}
\usepackage{amsthm}
\usepackage{amscd}
\usepackage{amsfonts}
\usepackage{mathtools}

\usepackage[
backend=biber,
style=alphabetic,
sorting=ynt
]{biblatex}


\newtheorem{proposition}{Proposition}
\newtheorem{remark}{Remark}


\newcommand{\RR}[0]{\mathbb{R}}
\newcommand{\EE}[0]{\mathbb{E}}
\newcommand{\bx}[0]{{\bf x}}
\newcommand{\bv}[0]{{\bf v}}
\newcommand{\balpha}[0]{{ \bf\alpha}}
\newcommand{\HamiltonJacobiPDE}[2]{
%#1: Value function, #2: independent running cost
#1_t (x,t) + H(x,D_x #1(x,t)) = #2
}
\newcommand{\HamiltonJacobiTerminalCondition}[2]{
%#1: Value function, #2: Terminal payoff
#1(x,T) = #2(x)
}


\newcommand{\DensityTransportPDE}[2]{
    % #1: Value function, #2: Density
    #2_t(x,t) + \Delta(H_p(x, D_x #1 (x,t)) #2(x,t)) = 0
}

\newcommand{\DensityTransportInitialCondition}[1]{
    % #1: Density
    #1(x,t_0) = #1_0(x)
}


\newcommand{\ReducedMfgStateEvolution}[2]{
\begin{equation}\label{reduced_mfg:state_evolution}
\begin{cases}
    \dot{ #1}_t = {#2}_t,\quad t\in (t_0,T],\\
    {#1}_{t_0} = #1_0.
\end{cases}
\end{equation}
}

\newcommand{\ReducedMfgPayoffFunctional}[5]{
\begin{equation}\label{reduced_mfg:payoff_functional}
J({\bf #2},#1; t_0) = \int_{t_0}^T \left(#3({\bf #2}_s,{\bf #1}_s) + #5 \right) ds + #4({\bf #1}_T),
\end{equation}}

\newcommand{\ReducedMfgHamiltonian}[4]{
\begin{equation}\label{reduced_mfg:HJ}
H(#1,#2) \coloneqq \sup_{#3} ( #2 \cdot #3 + #4(#3,#1)).
\end{equation}
}


\newcommand{\ReducedMfgStateHamiltonJacobiTerminalValueProblem}[3]{
\begin{equation}\label{reduced_mfg:state_evolution_hamiltonian}
\begin{cases}
    \HamiltonJacobiPDE{#1}{#2}, \\
    \HamiltonJacobiTerminalCondition{#1}{#3}.
\end{cases}
\end{equation}}

\newcommand{\ReducedMfgFeedbackStateEvolution}[3]{
\begin{equation}\label{reduced_mfg:feedback_state_evolution}
\begin{cases}
    {\dot {\bf #1}}_#2 = H_p({\bf #1}, D_x #3({\bf #1},#2)),\\
    {\bf #1}_{t_0} = #1.
\end{cases}
\end{equation}}


\newcommand{\ReducedMfgDensityTransportInitialValueProblem}[2]{
\begin{equation}\label{reduced_mfg:density_transport_pde}
\begin{cases}
    \DensityTransportPDE{#1}{#2},\\
    \DensityTransportInitialCondition{#2}
\end{cases}
\end{equation}}

\newcommand{\ReducedMfgMeanFieldGameSystem}[4]{
\begin{equation}\label{reduced_mfg:mean_field_game_pde}
\begin{cases}
    \HamiltonJacobiPDE{#1}{#4},\\
    \DensityTransportPDE{#1}{#2},\\
\end{cases}
\end{equation}}

\newcommand{\ReducedMfgMeanFieldGameBoundaryConditions}[4]{
\begin{equation}\label{reduced_mfg:mean_field_game_boundary}
\begin{cases}
    \HamiltonJacobiTerminalCondition{#1}{#3},\\
    \DensityTransportInitialCondition{#2}.
\end{cases}
\end{equation}}



\addbibresource{sample.bib}

\newcommand{\bx}[0]{{\bf x}}
\newcommand{\bv}[0]{{\bf v}}


\newtheorem{proposition}{Proposition}
\newtheorem{remark}{Remark}

\newcommand{\HamiltonJacobiPDE}[2]{
%#1: Value function, #2: independent running cost
#1_t (x,t) + H(x,D_x #1(x,t)) = #2
}
\newcommand{\HamiltonJacobiTerminalCondition}[2]{
%#1: Value function, #2: Terminal payoff
#1(x,T) = #2(x)
}


\newcommand{\DensityTransportPDE}[2]{
    % #1: Value function, #2: Density
    #2_t(x,t) + \Delta(H_p(x, D_x #1 (x,t)) #2(x,t)) = 0
}

\newcommand{\DensityTransportInitialCondition}[1]{
    % #1: Density
    #1(x,t_0) = #1_0(x)
}


\newcommand{\ReducedMfgStateEvolution}[2]{
\begin{equation}\label{reduced_mfg:state_evolution}
\begin{cases}
    \dot{\bf #1}_t = {\bf #2}_t,\quad t\in (t_0,T],\\
    {\bf #1}_{t_0} = #1.
\end{cases}
\end{equation}
}

\newcommand{\ReducedMfgPayoffFunctional}[5]{
\begin{equation}\label{reduced_mfg:payoff_functional}
J({\bf #2},#1; t_0) = \int_{t_0}^T \left(#3({\bf #2}_s,{\bf #1}_s) + #5 \right) ds + #4({\bf #1}_T),
\end{equation}}

\newcommand{\ReducedMfgHamiltonian}[4]{
\begin{equation}\label{reduced_mfg:HJ}
H(#1,#2) \coloneqq \sup_{#3} ( #2 \cdot #3 + #4(#3,#1)).
\end{equation}
}


\newcommand{\ReducedMfgStateHamiltonJacobiTerminalValueProblem}[3]{
\begin{equation}\label{reduced_mfg:state_evolution_hamiltonian}
\begin{cases}
    \HamiltonJacobiPDE{#1}{#2}, \\
    \HamiltonJacobiTerminalCondition{#1}{#3}.
\end{cases}
\end{equation}}

\newcommand{\ReducedMfgFeedbackStateEvolution}[3]{
\begin{equation}\label{reduced_mfg:feedback_state_evolution}
\begin{cases}
    {\dot {\bf #1}}_#2 = H_p({\bf #1}, D_x #3({\bf #1},#2)),\\
    {\bf #1}_{t_0} = #1.
\end{cases}
\end{equation}}


\newcommand{\ReducedMfgDensityTransportInitialValueProblem}[2]{
\begin{equation}\label{reduced_mfg:density_transport_pde}
\begin{cases}
    \DensityTransportPDE{#1}{#2},\\
    \DensityTransportInitialCondition{#2}
\end{cases}
\end{equation}}

\newcommand{\ReducedMfgMeanFieldGameSystem}[4]{
\begin{equation}\label{reduced_mfg:mean_field_game_pde}
\begin{cases}
    \HamiltonJacobiPDE{#1}{#4},\\
    \DensityTransportPDE{#1}{#2},\\
\end{cases}
\end{equation}}

\newcommand{\ReducedMfgMeanFieldGameBoundaryConditions}[4]{
\begin{equation}\label{reduced_mfg:mean_field_game_boundary}
\begin{cases}
    \HamiltonJacobiTerminalCondition{#1}{#3},\\
    \DensityTransportInitialCondition{#2}.
\end{cases}
\end{equation}}



\title{Economic Models and Mean-Field Games Theory - Summary}
\author{Felipe Antunes}
\date{August 2022}

\begin{document}

\maketitle
\section{Introduction}
Mathematical methods allow for testable models in modern economic theory. It enables authors to formulate and solve models in a unified language.

Mean-field games - Methods and techniques to study differential games with a large population of rational players. Typically formulated in terms of partial differential equations, namely a transport or Fokker-Plank equation for the distribution coupled with a Hamilton-Jacobi equation.

Applications in economics, sustainable development.

The importance with regards to economic theory stems from the fact that it allows a systematic formalization ogf rational expectations theory and heterogeneous agent models.

Existence and regularity of solutions is an important research area.


\section{Deterministic Models}

\subsection{First-order mean-field games}

\subsubsection{Deterministic optimal control problems}\label{deterministic_optimal_control}

Single agent whose state is denoted by $x_t \in \RR^d$, for $t_0 \leq t\leq T$, where $T > 0$ is arbitrarily fixed. The case $T< \infty$ is referred to as the finite horizon optimal control problem.

$$\dot x_t = f(v_t,x_t),$$

where $f: \RR^m \times \RR^d \mapsto \RR^d$ is a given function and $v_t \in \RR^m$ is a control. Let $\mathcal{W} \coloneqq \left\{ v : [t_0, T] \mapsto W, v \in X \right\}$, where $X$ is a suitable function space.

Agent's preferences are expressed through a utility functional $J(u_t, x_t; t)$. For $t \in [t_0,T)$ this functional is given by 

$$
J(u,x,t) = \int_t^T u(v_s, x_s) ds + \Psi(x_T),
$$
where $u : \RR^m\times \RR^d \mapsto \RR$ is the instantaneous utility (or running cost) and $\Psi : \RR^d \mapsto \RR$ is the terminal payoff.

In the case $T = \infty$, we introduce a discount rate $\beta > 0$ and consider
$$
J_\beta (u,x;t) = \int_t^\infty e^{-\beta (s - t) u(v_s, x_s)} ds 
$$

The focus of this section will be a variant of the last problem, namely:

$$
J_\alpha (u,x;t) = \int_t^\infty e^{-\alpha (s - t) u(v_s, x_s)} ds + e^{-\alpha (T -t)} \Psi(x_T) 
$$

The single agent seeks to maximize $J_\alpha$ over all possible controls.
We define the value function $v(x,t)$ as

$$
V(x,t) = \sup_{v\in \mathcal{W}} J_\alpha (v, x ; t).
$$
This function is the maximum possible payoff for the agent at state $x$ in time $t$.

Assumptions on $f$ (for the problem to be addressed rigorously):
$$f \in C(\RR^m \times \RR^d, \RR^d),\, |f(v, x_1) - f(v, x_2)| \leq C_\theta |x_1 - x_2|, x_1, x_2 \in \RR^d, v\in \RR^m, |v| \leq \theta$$

If the Value function is of class $C^1(\RR^d \times [t_0,T])$, it solves  a nonlinear partial differential equation known as the Hamilton-Jacobi equation:
$$V_t(x,t) - \alpha V(x,t) + H(x,D_x V(x,t)) = 0,$$
where the Hamiltonian H is defined as 
$$H(x,p) = \sup_{v\in W} [p \cdot f(v,x) + u(v,x)],$$
that is, the Legendre transform of the running cost $u$.
The optimal control $v_t^*$ is given by 
$$ H_p (x^*, D_x V(x^*,t)) = f(v^*, x^*)$$

\subsubsection{Dynamic Programming Principle}

\begin{proposition}[Dynamic programming principle]
    Let $r \in (t_0, T)$ be fixed and assume that $V$ is a value function. Then we have
    $$
    V(x,t) = \sup_{v\in \mathcal{W}} \left( \int_r^T e^{-\alpha (s-r)} u(v_s,x_s) ds + e^{-\alpha (r - t) V(x_r,r)} \right).
    $$
\end{proposition}

\begin{proof}
    To do
\end{proof}

\subsubsection{Verification Theorem for Hamilton-Jacobi}
\begin{proposition}[Verification Theorem]
    Let $U \in C^1(\RR^d \times [t_0, T])$ be a solution to 
    $$
    U_t(x,t) - \alpha U(x,t) + H(x, U_x(x,t)) = 0\, \text{(HJ)}
    $$
    where 
    $$
    H(x,p) = \sup_{v \in W} [p \cdot f(v,x) + u(v,x)],
    $$
    with terminal condition $U(x,T) = \psi(x)$. Then
    $$U(x,t) \geq V(x,t), \forall (x,t) \in \RR^d \times [t_0,T].$$
    Moreover, if there exists a control $v^*$ such that
    $$
    H(x^*_t, U_x (x^*_t, t)) = U_x(x^*_t, t)\cdot f(v^*_t, x^*_t) + u(v^*_t, x^*_t),
    $$
    That is, $v^*$ gives an equality in the Hamiltonian, then $v^*$ is an optimal control for the control system and $U(x,t) = V(x,t)$.
\end{proposition}

\begin{proof}
    To do
\end{proof}

\subsubsection{Pontryagin Maximum Principle}

The PMP gives a necessary condition for an optimal control in terms of an adjoint variable $q: \RR \mapsto \RR^d$. The adjoint variable satisfies:
\begin{equation}\label{pmp:adjoint_eq}
    \frac{d}{ds} \left( e^{-\alpha(s-t)} {\dot q}_j(s) \right) = -e^{-\alpha (s - t)} \sum_{i = 1}^d \frac{\partial}{\partial x_j} f_i (v^*_s, x^*_s) q_i(s) - e^{-\alpha (s - t)} \frac{\partial}{\partial x_j} u(v^*_s, x^*_s),
\end{equation}

with $q(T) = \nabla \psi(x^*_T)$.

\begin{proposition}[Pontryagin Maximum Principle]
    Let $v^*$ be an optimal control for the optimal control problem and assume that $q$ is a solution for the terminal value problem \eqref{pmp:adjoint_eq}. Then,
    \begin{equation}\label{pmp:transversality_condition}
    q(s) \cdot f(v^*_s, x^*_s) + u(v^*_s, x^*_s) = H(x^*_s, q(s)).
    \end{equation}
\end{proposition}

\begin{proof}
    To do.
\end{proof}

Condition \eqref{pmp:transversality_condition} is called a \textit{tranversality condition}.

\subsection{Transport Equation}\label{transport_eq}
Consider a population that is governed at the single agent level by
\begin{equation}\label{transport_eq:ode}
\begin{cases}
    \dot x_t = b(x_t,t), \quad t > t_0,\\
    x_{t_0} = x.
\end{cases}
\end{equation}
where $b$ is a Lipschitz vector field. The vector field induces a flow that trasports the population's density along time (pushforward). The evolution of the density is described be the \textit{transport equation}, a first order pde:  

\begin{equation}\label{transport_eq:pde}
\rho_t + \Delta(b(x,t) \rho) = 0.
\end{equation}

\begin{proposition}[Properties of solutions]    
Let $\rho$ be a solution of \eqref{transport_eq:pde} with initial condition $\rho(x,t_0) = \rho_0(x)$, for some probability density $\rho_0 \in \mathcal{C}_c^\infty (\RR^d).$ Then we have:
\begin{itemize}
    \item Positivity of solutions: $\rho(x,t) \geq 0$ for all $t> t_0$,
    \item Mass conservation: $\int_{\RR^d} \rho(x,t) dx = \int_{\RR^d} \rho_0(x) dx = 1$ for all $t > 0$.
\end{itemize}
\end{proposition}

\begin{proof}
    To do.
\end{proof}

Define the flow in $\RR^d$ given by $b$ by $\Phi_t$. This flow maps $x\in \RR^d$ to the solution of \eqref{transport_eq:ode} at the instant $t > t_0$, with initial condition $x$.

We can use the operator $\Phi_t$ to define a family of measures $\theta(x,t)$ over $\RR^d$ from an initial measure $\theta_0$ by 
\begin{equation}\label{transport_eq:family_of_measures}
\int_{\RR^d} \phi(x) \theta(x,t) dx = \int_{\RR^d} \phi(\Phi_t(x)) \theta_0 dx    
\end{equation}

This measure satisfies the transport PDE in the distribution sense. 
\begin{proposition}[Solutions in the distribution sense]
Let $\theta$ be the measure defined in \eqref{transport_eq:family_of_measures}. Assume that the vector field $b$ is Lipschitz continuous and denote by $\Phi_t$ the flow corresponding to \eqref{transport_eq:ode}. Then
\begin{equation}\label{transport_eq:pde_for_measures}
    \begin{cases}
        \theta_t(x,t) + \Delta(b(x,t) \theta(x,t)) = 0, \quad (x,t) \in \RR^d \times [t_0,\infty),\\
        \theta(x,t_0) = \theta_0(x), \quad x\in \RR^d.
    \end{cases}
\end{equation}
in the distributional sense.
\end{proposition}

\begin{proof}
    To do.
\end{proof}

\begin{proposition}[Dirac delta solution]\label{transport_eq:prop:dirac_delta_solution}
Assume that \eqref{transport_eq:ode} holds and suppose that $b$ is Lipschitz continuous. Then, $\delta_{\Phi_t(x)}$ solves \eqref{transport_eq:pde_for_measures} in the distributional sense.
\end{proposition}

\begin{proof}
    To do.
\end{proof}

\begin{remark} The converse of \eqref{transport_eq:prop:dirac_delta_solution} also holds.
\end{remark}

\subsection{Reduced Mean-field games}\label{reduced_mfg}
Among the main motivations for studying MFG are the connections with the theory of $N$-player differential games. A solution of the MFG formalizes the limit case of Nash equilibrium when $N \to \infty$ for $N$-person differential games.

Some reduced mean-field games models can be rigorously derived as the limit of equations characterizing an $N$-player differential game. This subsection presents a heuristic derivation.

Consider first the single-agent optimal control problem. Suppose that the state of an agent at time $t$ is characterized by the vector $\bx_t \in \RR^d$. In addition, assume that the following differential equation describes the state $\bx_t$:
\ReducedMfgStateEvolution{x}{v}
where $\bv_t$ is a control and $x\in\RR^d$ is a given initial condition. This agent seeks to maximize the functional:
\ReducedMfgPayoffFunctional{x}{v}{u}{u_T}{g[\rho]}
where $u$ is a Lagrangian, $u_T$ is a terminal payoff and $g[\rho]$ is a term to be made precise later. 

Define
\ReducedMfgHamiltonian{x}{p}{v}{u}
We know from Section \eqref{deterministic_optimal_control} that under some regularity assumptions, the value function $V$ associated with the optimal control problem \eqref{deterministic_optimal_control:optimal_control_problem} 
solves the Hamilton-Jacobi equation in $\RR^d \times [t_0,T)$:
\ReducedMfgStateHamiltonJacobiTerminalValueProblem{V}{g[\rho]}{u_T}
Moreover, the optimal control $\bv^*$ is given in feedback form by
$$\bv^* = H_p(\bx, D_x V(\bx, t)).$$
The state of a rational agent will evolve according to \eqref{reduced_mfg:state_evolution} with the control $\bv$ equal to the feedback optimal control $\bv^*$.

Consider then a population of rational and indistinguishable agents. The distribution of agents is given by a probability measure $\rho$. This population of agents is subjected to the same optimal control problem. Therefore, the state of each individual is driven by
\ReducedMfgFeedbackStateEvolution{x}{s}{V}

The function $g$ encodes the dependence of the agent's cost functional on the density of the entire population. This implies that the optimization problem faced by the agents depends on the evolution of $\rho$. Conversely, the evolution of $\rho$ depends on a vector field determined by $V$, which is derived from the control problem of the agents.\footnote{Chicken and the egg problem.}

From Section \eqref{transport_eq}, we know that if $\bx$ satisfies \eqref{reduced_mfg:feedback_state_evolution}, then the evolution of the density $\rho$ in time satisfies a transport equation in $\RR^d \times (t_0, T]$:
\ReducedMfgDensityTransportInitialValueProblem{V}{\rho}

The mean-field game system associated with this problem is the system of PDE's:
\ReducedMfgMeanFieldGameSystem{V}{\rho}{u}{g[\rho]}
coupled with the boundary conditions
\ReducedMfgMeanFieldGameBoundaryConditions{V}{\rho}{u}{g[\rho]}

\section{Some Simple Economic Models}
Simplified model that depict the key techniques
\begin{itemize}
    \item Solow model for capital accumulation - Not strictly MFG - no interaction between agents.
    \item Planned Economy Solow model - Example of Mean Field control.
    \item Deterministic Aiyagari-Bewley-Hugget model - wages and capital accumulation.
\end{itemize}

\subsection{Economic Models}
\subsubsection[]{Agents and their states}

\begin{itemize}
    \item Microeconomic agents - Small participants in the econoy. Individually, cannot influence the economic
    outcomes - consumers, works, small firms.
    \item Macroeconomic agents - Agents able to impact directly the economy - central banks, large firms, governments.
    \item Microeconomic variables - Microeconomic agents characterized by individual state variables,
    such as wealth, capital, wages. Probability measure $\rho$ describes the distribution of agents
    in the economic state space. Heterogeneity is the hallmark of heterogeneous agent models - captured by $\rho$.
    \item macroeconomic variables - a global quantity of the economic system. Exogenous variables such as 
    taxes, external demands for goods, or variables determined by equilibrium conditions such as relative prices.
\end{itemize}

\subsection[]{Dynamics and controls}

\begin{itemize}
    \item Controls (actions, strategies) - quantities that agents are allowed to choose the value for. 
    Either micro or macro - for the respective category of agent.
    \item Constitutive relations - Relations between various quantities in the economy. Model dependent.
    \item Dynamics - Laws describing the evolution of the agents' state. Can be deterministic or stochastic
     (diffusions, jump processes). 
\end{itemize}

\subsection{Preferences}

Assumption of rationality - agents choose controls in order to maximize some functional. The functional 
encondes the preferences of the agents with respect to the variables of the model.
For example, microeconomic agents prefer to consume more than less, central banks aim to control prices
while promoting growth and capital accumulation.

For microeconomic agents, preferences are formalized through an instantaneous utility function.
Properties such as monotonicity and concavity encode qualitative properties of the model -
ie concavity models diminishing increase in satisfaction for increments in consumption.
For macroeconomic agents, an instantaneous welfare function is chosen.

Choices are made in a particular time span - intertemporal counterparts of the utility and welfare
functions.

\subsection{Equilibrium conditions}

Equilibrium conditions are constraints relating the many quantities involved in the model. 
They formalize the intuitive notions behind economic processes to be modelled.

For example, in a closed economy, the total investment must equal the total production of capital goods.

\subsection{Solow Model with Heterogeneous Agents}

Heterogeneous agent capital cacumulation model - Illustrates two distinct economies:
A free market economy where the price level is determined by a price clearing condition,
and a centrally planned economy in which a central planner sets the share of production consumed
by each agent.

The basic elements of the model are:

\begin{itemize}
    \item Microeconomic agents: Agents are characterized by their levels of capital $k_t \geq 0$. The 
    distribution of agents in the state space is given by $\rho = \rho(k,t)$.
    \item Constitutive relations: The model has a resource constraint entailed by a production function
    $f = f(k)$. The production function gives the output $f(k)$ of an a agent with capital $k$ and 
    encodes the technology of the economy.

    We assume that $f'(k) > 0$, for $k >0 $, and 
    $$
    \lim_{k \to 0} f'(k) = +\infty,\quad \text{and} \quad \lim_{k\to \infty} f'(k) = 0.
    $$
    The limits formalize the notion of marginal productivity of capital.

    Production of agent with capital $k_t$ is divided between an amount $c_t$ of consumption and $i_t$ of 
    investment. This is equivalent to the following resource constraing:
    $$
    f(k_t) = c_t + i_t,
    $$ 
    where $c_t, i_t \geq 0$.
    \item  Dynamics: capital accumulation - accumulation is governed by the following ode:
    $$
    \dot k_t = i_t - \delta k_t,
    $$
    where $0 < \delta < 1$ is the capital depreciation rate of the economy, and $i_t$ is the investment.
\end{itemize}

We'll solve this model in two settings: a free market economy where agents choose their consumption level
in order to maximize some functional $J = J(c,k,t)$, and a planned economy in which consumption is chosen
by a central planner that maximizes a welfare function of the economy's consumption level.

\subsubsection{Free-market economy}

Agents are free to allocate their production $f(k_t)$ between consumption and investment, subject to 
$$ f(k_t) = c_t + i_t. $$

\begin{itemize}
    \item Microeconomic controls (or actions) - Agents control their level of consumtption $c_t$.
    The agents select the control by maximizing the utility functional (see next item), subject to the 
    constraints $c_t \geq 0, i_t \geq 0$. We conclude that $0\leq c_t \leq f(k_t)$.
    \item Agent's preferences - We model agents' preferences by a utility function $u = u(c,k)$, with descreasing
    marginal increments i.e. the map $(c,k) \mmapsto u(c,k)$ is increasing and concave. The rational 
    behaviour of agents is modelled by assuming that agents' maximize 
    their aggregate discounted utility over time:
    $$
    J(c_t, k_t, t) = \int_t^\infty e^{-\alpha (s - t)} u(c_s, k_s) ds.
    $$

    This implies that agents choose their consumption level by solving the following optimal control
    problem:
    \begin{equation}
        \begin{cases}
            max_{c_{(\cdot)} \in \mathcal{C}} J(c_t, k_t, t),\\
            \text{s.t.} \, \dot k_t = i_t - \delta k_t.
        \end{cases}
    \end{equation}

    If the value function $V(k,t) = \sup_{c_t} J(c_t, k_t, t)$ is regular enough, it solves the HJ equation:
    $$
    V_t - \alpha V + H(k,V_k) = 0,
    $$
    where $H(k, q_k)$ is given by:
    $$
    H(k, q_k ) = \sup_{0 \leq c \leq f(k)} ((f(k) - c - \delta k) q_k + u(c,k))
    $$
    and the optimal control $c^*$, if it exists, is given in feedback form by
    $$
    H_q(k,q) = f(k) - c^* - \delta k.
    $$
\end{itemize}

The density of agents in the state space evolves according to a transport equation. 
{\bf The assumption of rationality implies that every agent acts according to the optimal control.}
The evolution of density $\rho$ is given by 
$$
\rho_t(k,t) + (H_q(k,V_k) \rho(k,t))_k = 0
$$

Finally, the Solow growth model for a free-market economy can be written as:

\begin{equation}
    \begin{cases}
        V_t - \alpha V + H(k,V_k) = 0,\\
        \rho_t(k,t) + (H_q (k,V_k) \rho(k,t))_k = 0.
    \end{cases}
\end{equation}

\subsubsection{Planned Economy}
This time, suppose that agents have a production function $f(k)$ but they are not allowed to allocate
freely between consumption and investment.
\begin{itemize}
    \item Constitutive relations: A planner controls the fraction $\lambda_t$ of the production
    that can be invested or consumed:  $i_t = \lambda_t f(k_t)$, $c_t = (1 - \lambda_t) f(k_t)$.
    \item Macroeconomic controls: The central planner controls the rate $\lambda_t \in [0,1]$ - the same
    for all players.
    \item Transport of the agents' density: 
    $$
    \rho_t (k,t) + ( (\lambda_t - \delta) f(k) \rho(k,t) )_k = 0.
    $$
    \item Macroeconomic variables: aggregate production $F_t = \int f(k) \rho(k,t)$.
    \item Macroeconomic preferences: Central planner maximizes a welfare function that measures consumption.
    the planner has an instantaneous welfare function $U$ and a positive discount factor $\alpha$:
    $$
    \int_t^{+\infty} e^{-\alpha (s-t)} U( (1-\lambda_s)F_s )ds.
    $$
    
    \item The welfare functional together with the tranport equation pose a mean-field control problem.
    In this sense, $\rho$ is a state variable with controled dynamics given by the transport equation.
    We seek to maximize the welfare functional that depends on $\rho$ through the aggregate production.
\end{itemize}

\subsubsection{Aiyagari-Bewley-Hugget model}

The model describes an economy where agents are heterogeneous in wealth and wage. The interest rate
is the single macroeconomic variable, and it is determined by a zero net supply condition.

\begin{itemize}
    \item Microeconomic agents: At each instant $t \geq t_0$, agents in the economy are characterized by
    their income and wealth levels: $z_t$ and $a_t$.
    
    The agents' are distributed in the state space according to a probability density $\rho = \rho(a,z,t)$.

    \item Macroeconomic variables: The interest rate of the economy is denoted by $r_t$. This is the 
    rate at which agents can borrow or lend wealth (no financial intermediaries).

    \item Constitutive relations: We assume the evolution of the wage of each agent follows a mean-reverting
    dynamic $\dot z = - (z_t - \bar z)$, where $\bar z$ represents a reference wage level. To simplify
    the presentation, set $\bar z \equiv 1$. This choice for constitutive relation is rather arbitrary.

    \item Microeconomic actions: agents are assumed to choose their consumption levels $c_t$ for 
    $t_0 \leq t \leq T$.

    \item Microeconomic dynamic: the state variables are governed by a system of odes:
    \begin{equation}
        \begin{cases}
            \dot a_t = z_t - c_t + r_t a_t, \\
            \dot z_t = - (z_t - 1).
        \end{cases}
    \end{equation}

    \item Microeconomic preferences: We assume that agents are rational i.e. they maximize a set of
    preferences over the variables of the model aggregated over time.
     We assume these preferences to be modeled by an
    instantaneous utility function $u = u(c_t,a_t, z_t, r_t)$. 
    The discounted aggregated future utility $J$ is given by:
    $$
    J(c,a,z; t) = \int_t^\infty e^{- \alpha (s - t)} u(c_s, a_s, z_s, r_s) ds.
    $$
    The value function is given by
    $$
    V(a,z,t) = \sup_{c_t} J(c_t, a_t, z_t; t).
    $$
    If $V$ is regular enough, it is the solution to the following HJ equation:
    $$
    V_t - \alpha V + H(a,z,r,V_a,V_z,t) = 0,
    $$
    where the Hamiltonian $H$ is given by
    $$
    H(a,z,r, q_a, q_z) = \sup_{c} (q_a \dot a + q_z \dot z + u).
    $$
    The optimal control $c^*_t$ (optimal consumption level) is given in feedback form by:
    \begin{equation}
        \begin{cases}
            H_{q_a}(a,z,r,q_a,q_z) = z_t - c_t^* + r_t a_t,\\
            H_{q_z} (a,z,r,q_a,q_z) = -(z_t - 1).
        \end{cases}
    \end{equation}
    
    \item Equilibrium conditions: We assume that agents in the economy can borrow and lend at the current
    interest rate $r_t$. To close the model, we must assume that each dollar borrowed by an agent was
    saved by another one. This condition is formalized by the {\it zero net supply condition}:
    $$
    \int a \rho(a,z,t) da dz = 0;
    $$
    This condition implies that $r_t$ depends on $\rho$ ($r_t$ is implict in $a$) -> the interest rate encodes
    the dependence of $V$ on the density of the population.

    \item Transport of the agents' density: The density $\rho(a,z,t)$ evolves according to
    the transport equation:
    $$
    \rho_t (a,z,t) + (H_{q_a} \rho)_a + (H_{q_z} \rho)_z = 0.
    $$
    Using the formula for the optimal control in feedback form (i.e. assuming agents behave rationally):
    $$
    \rho_t(a,z,t) + (( z - c^* + ra) \rho)_a - ( (z - 1) \rho )_z = 0.
    $$
    We can finally write the deterministic Aiyagari-Bewley-Hugget model as the following mean-field game
    system:
    \begin{equation}
        \begin{cases}
            V_t - \alpha V + H(a,z,r,V_a,V_z) = 0,\\
            \rho_t (a,z,t) + (( z - c^* + ra) \rho)_a - ((z-1) \rho)_z = 0.
        \end{cases}
    \end{equation}
    together with initial condtions for the density $\rho$.

    \item Analysis of the equilibrium condition: 
        Sketch calculations to understand what he is doing.
\end{itemize}

\section{Economic Growth and MFG - Deterministic Models}

    Some models of economic growth with heterogeneous agents:
    \begin{itemize}
        \item Wealth and capital accumulation, capital dependent production function. 
        Model is a MFG where the coupling is determined by an economic equilibrium condition.
        \item Central bank controlling interest rate. The problem becomes an optimal control
        of a mean-field game (principal agent problem).
        \item Trade imbalances and international trade. The model becomes an infinite
        dimensional game between two agents, whose dynamics is given by two mean-field games
        (a game between two principal agents).
        \item Price impacts in regulated markets. This is a mean-field game that depends
        on a price-impact function (i.e. import duties). 
    \end{itemize}
   
\subsection{A growth model with heterogeneous agents}
    A simple wealth and capital accumulation problem, with capital-dependent production.

\begin{itemize}
    \item Microeconomic Agents: agent has amount $a_t$ of consumer goods and $k_t$
    units of capital at time $t$. Consumer goods are used as the numerary of the economy
    in the walrasian sense.

    Both the consumer goods $a_t$ and the capital $k_t$ have natural constraints. 
    Borrowing constraints correspond to the inequality $a_t \geq a_0$, for some $a_0 \leq 0$.
    Capital is assumed to be non-negative $k_t \geq 0$. These constraints will be implemented
    as soft constraints through a utility function.

    The microeconomic distribution function $\rho(a,k,t)$ determines the distribution
    of agents in the state space. We suppose that $\rho \geq 0$ is normalized, so that
    $$
    \int_{\mathbb{R}^2} \rho(a,k,t) da\, dk = 1,\quad \forall t\geq 0.
    $$
    \item Macroeconomic variables: The only macroeconomic variable is the relative price
    of capital $p_t$. We refer to 
    $$
    W_t = \int (a_t + p_t\, k_t) d\rho(a,k,t)
    $$
    as the total wealth of the economy.
    \item Constitutive relations: two constitutive relations:
    First, a production function $F(k,p)$ that gives the total value (measured in consumer goods) 
    of consumer $\Theta(k,p)$ and capital goods $\Xi(k,p)$ produced by an agent with capital $k$
    at a price level $p$. In particular, $$ F(k,p) = \Theta(k,p) + p\Xi(k,p) $$.
    The functions $\Thete$ and $\Xi$ capture the fact that agents can change their production
    from consumer to capital goods and vice-versa.
    Second, depreciation of capital is governed by a function $g(k,p)$.

    \item Microeconomic actions (controls): 
    Agents control their consumption and investment level $c_t$ and $i_t$ at time $t$.

    \item Microeconomic dynamics:
    The stock of consumer goods evolves by $$\dot a_t = - c_t - p_t i_t + F(k_t, p_t).$$
    The stock of capital evolves according to $$\dot k_t = g(k_t,p_t) + i_t.$$ 

    \item Microeconomic preferences: agents have preferences about
    their consumption, investment, stocks of consumer and capital goods, and the price level
    of the economy.
    Preferences are represented by an instantaneous utility function
    $u = u(c_t,i_t,a_t,k_t,p_t)$, and agents try to maximize the intertemporal counterpart
    to $u$ (assumption of rationality):
    $$
    V(a,k,t) = \sup_{c_t,i_t} \int_t^\infty e^{-\alpha(s-t)} u(c_s,i_s,a_s,k_s,p_s) ds
    $$
    \item Optimal control problem: 
    Each agent faces an optimal control problem. $V(a,k,t)$ is its value function. 
    If $V$ is regular enough, it solves the following HJB equation:
    $$
    V_t(a,k,t) - \alpha V(a,k,t) + H(a,k,p,V_a,V_k) = 0,
    $$ 
    where $H$ is given by 
    $$
    H(a,k,p,V_a,V_k) = \sup_{c,i}\left( V_a {\dot a }+ V_k {\dot k} + u \right). $$
    Moreover, the optimal controls $c_t^*, i_t^*$ are given in feedback form Bewley
    $$
    \begin{cases}
        H_{q_a} (a_t, k_t, p_t, V_a, V_k) = - c_t^* - p_t i^*_t + F(k_t,p_t),\\
        H_{q_k} (a_t, k_t, p_t, V_a, V_k) = g(k_t, p_t) + i_t^*
    \end{cases}
    $$
    \item Transport of the agents' population:
    As the agents' evolution is given by differential equations, the evolution in time of the
    population density is given by a transport equation. Let $\rho(a,k,t)$ be the density of the agents.

    The assumption that agents optimize their preferences is central to the optimal control problem
    - however, note that the controls (consumption and investment) affect the evolution of the agents.

    Rationality implies that individuals consume the optimal consumption $c^*$ and invest the optimal
    investment $i^*$. Therefore, the evolution of the population is driven by $(\dot a, \dot k)$ evaluated
    at these values. Therefore, we have the following transport equation for $\rho$:
    $$
    \rho_t + ( (- c^* - p_i i^* + F(k)) \rho )_a + ( (g(k,p_t) + i^*) \rho )_k = 0
    $$
    Using the feedback form for the controls, we can write the previous equation as
    $$
    \rho_t + (H_{q_a} \rho)_a + (H_{q_k} \rho)_k = 0
    $$
    \item Equilibrium conditions: The HJ and transport equations are coupled by a market-clearing condition.
    This condition requires that the aggregate production of capital goods match the aggregate investment,
    and is given by
    $$
    \int i d\rho(a,k,t) = \int \Xi(k,p) d\rho(a,k,t)
    $$
    This equilibrium condition determines the price level $p_t$.

    \item A mean-field game model: The mean-field game formulation of the growth model is given by
    the coupling of the $HJ$ equation with the transport equation, along with the equilibrium condition.
    $$
    \begin{cases}
        V_t(a,k,t) - \alpha V(a,k,t) + H(a,k,p,V_a,V_k) = 0,\\
        \rho_t + (H_{q_a} \rho)_a + (H_{q_k} \rho)_k = 0.
    \end{cases}
    $$
    An initial condition is given for $\rho$.
\end{itemize}

\subsection{A growth model with a macroeconomic agent}

Building upon the previous section, we introduce a growth model with a central bank acting as 
a macroeconomic agent, controling the interest rate of the economy. Its goal is to balance
price stability with economic growth.

\begin{itemize}
    \item Microeconomic agents: As in the previous section, characterized by
    consumer good stock $a_t$ and capital $k_t$.
    \item Macroeconomic agent: The single macroeconomic agent is a central bank.
    Its state is determined by a quantity $A_t$, that represents its assets, and
    the distribution of microeconomic agents $\rho(a,k,t)$.
    \item Macroeconomic variables: As before, the price level $p_t$ of the economy is a macroeconomic
    variable. The economy has an interest rate $r_t$, which is set by the central bank. Therefore, it 
    is a control variable for this macroeconomic agent.
    \item Constitutive relations: The same as the previous section: production and depreciation functions.
    \item Microeconomic actions (controls): The same as the previous section: consumption and investment levels.
    \item Microeconomic Dynamics:
    The stock of consumer goods is driven by 
    $$
    \dot a_t = r_t a_t - c_t - p_t i_t + F(k_t, p_t).
    $$
    Note that this equation depends explicitly on the interest rate of the economy, $r_t$.
    The stock of capital varies according to
    $$
    \dot k_t = g(k_t, p_t) + i_t.
    $$
    Note that introducting a central bank in the economy does not influence directly
    the dynamics of capital accumulation, for the interest rate is not included in the capital ODE.
    However, because it affects the decision of consumption and investment, the impact of $r_t$ on
    the capital accumulation is ecoded in the equilibrium condition of the model.
    \item Macroeconomic actions (controls): The central bank intervenes in the economy by setting
    its interest rate $r_t$.
    \item Microeconomic preferences: We assume the instantaneous utility function of the microeconomic
    agents also depend on the interest rate: $u = u(c_t, i_t, a_t, k_t, p_t, r_t)$. It amounts to 
    $$
    V(a,k,t) = \sup_{c_t,i_t} \int^\infty_t e^{-\alpha(s-t)} u(c_s, i_s, a_s, k_s, p_s, r_s) ds
    $$ 
    \item Equilibrium condition: the same as last section.
    \item Optimal control problem: The microeconomic agents' optimal control problem is described by 
    $$
    V_t(a,k,t) - \alpha V(a,k,t) + H(a,k,p,r,V_a,V_k) = 0,
    $$
    where H is given by $H(a,k,p,r,V_a,V_k) = \sup_{c,i} \left( V_a {\dot a} + V_k {\dot k} + u \right)$
    The optimal contols in feedback form are given by
    $$
    \begin{cases}
        H_{q_a} (a_t, k_t, p_t, r_t, V_a, V_k) = r^*_t a_t - c^*_t - p_t i^*_t + F(k_t,p_t),\\
        H_{q_k} (a_t, k_t, p_t, r_t, V_a, V_k) = g(k_t, p_t) + i^*_t,
    \end{cases}
    $$
    Note that the microeconomic controls are distinct from those in the previous section. Moreover,
    a macroeconomic policy, when it exists, depends on the microeconomic optimal controls and conversely.
    \item Transport of the agents' population:
    Again, the dynamics of the agents' yield a transport equation of the density $\rho(a,k,t)$:
    $$
    \rho_t + (()r^* a_t - c^* - p_t i^* + F(k,p_t)\rho)_a + ( (g(k,p_t) + i^*)\rho )_k = 0
    $$
    This transport equation coupled with the optimization problem leads to the following mean-field game system:
    $$
    \begin{cases}
        V_t(a,k,t) - \alpha V(a,k,t) + H(a,k,p_t,r_t,V_a,V_k) = 0,\\
        \rho_t + (H_{q_a} \rho)_a + (H_{q_k} \rho)_k = 0.
    \end{cases}
    $$
    \item Macroeconomic dynamics: 
    The ammount of consumer goods $a_t$ that is neither consumed nor invested earns an interest rate $r_t$
    set by the central bank. Therefore, the position of the central bank $A_t$ is governed by
    $$
    \dot A_t = -r_t \int a \rho(a,k,t) da\, dk
    $$

    \item Macroeconomic welfare:
    The macroeconomic agent wants to control two quantities of the model: the price level of the economy
    $p_t$ and the total wealth of the system 
    $$
    W_t = \int a \rho(a,k,t) + p_t \int k \rho(a,k,t)
    $$
    The central bank maximizes the welfare function of the economy given by $U = U(A_t, W_t, p_t)$.
    The optmization problem faced by the central bank is formulated in terms of the intertemporal counterpart
    of $U$:
    $$
    \sup_{r_t} \int^\infty_t e^{-\beta (s-t)} U(A_s,W_s,p_s) ds.
    $$
    When it exists, the optimal control $r^*_t$ will be referred to as the macroeconomic policy.
    The function $U$ can be taken to be general enough to accomodate various formulations,
    i.e. keeping the price level, sustaining economic growth.
    \item Macroeconomic policy: 
    The macroeconomic policy is determined by the control problem given by the intertemporal counterpart
    of $U$, where the control $r_t$ affects the macroeconomic state $(A,\rho)$ through the dynamic of 
    the macroeconomic agent and the MFG. Therefore, the optimization problem of the macroeconomic agent can 
    be regarded as the control of an ODE and a system of PDEs.
\end{itemize}

\subsection{Economic growth in the presence of trade imbalances}

    Model of the economy with trade imbalances - this means the market clearing condition fails to hold.
    Trade imbalances can come from various forces - we'll study the case of a closed economy and
    the case of international trade.

\subsubsection{Single economy with trade imbalance and regulated prices}

    We say that a trade imbalance happens in the economical system if the price level is not determined
    by the market clearing condition. In a single economy, this can have multiple causes. 
    For example, price level $p_t$ may be subjected to distortions - i.e. regulated prices - or
    follow a dynamic prescribed a priori.

    In this subsection, we'll assume the price is regulated by a central planner (or any other 
    external entity to the economy).

\begin{itemize}
    \item Equilibrium condition: Trade imbalances quantify the difference between the
    total investment in the economy and the production of capital goods. 
    Given an extrinsic price level $p_t$, we can quantify the trade imbalance by 
    $$
    E_t = \int i d\rho (a,k,t) - \int \Xi(k,p) d\rho (a,k,t).
    $$
    The case $E_t > 0$ represents an excess of demand of capital goods, whereas
    $E_t < 0$ stands for an excess in supply. Therefore, for $E_t \neq 0,$ the
    market does not clear.
    \item Microeconomic state variables: Stock of consumer goods and capital accumulation 
    evolve as before.
    \item Microeconomic preferences: Agents in the economy have preferences over
    the trade imbalance $E_t$. These are encoded by the instantaneous utility function
    $u = u(c_t, i_t,a_t, k_t, p_t, r_t, E_t).$ Hence, the intertemporal counterpart of $u$ becomes
    $$
    V(a,k,t) = \sup_{c_i, i_t} \int_t^\infty e^{-\alpha (s-t)} u(c_s, ..., E_s) ds.
    $$
    \item Optimal control problem: As before.
    \item Transport of the agents' density: As before.
    \item Macroeconomic preferences: The preferences of the central bank are
    also affected by the trade imbalance. The welfare function becomes 
    $U = U(A_t,W_t,p_t,E_t)$, where $W_t$ is the total wealth
    $$
    W_t = \int a \rho(a,k,t) + p_t \int k \rho (a,k,t)
    $$
    And the optimal control problem of the macroeconomic agent is formulated in
    terms of the intertemporal counterpart of $U$
    $$
    \max_{r_t} \int^\infty_t e^{-\beta (s - t)} U(A_s, W_s, p_s, E_s) ds.
    $$
    \item Macroeconomic policies: The macroeconomic policy is determined by the
    control problem above, where the control $r_t$ affects the 
    macroeconomic state $(A,\rho)$ through the dynamic composed of the macroeconomic
    dynamics defined in the last section and the MFG. Therefore, the control problem
    of the macroeconomic agent can be regarded as the control of an ODE and a system
    of PDEs.
\end{itemize}

\subsubsection{Two countries economy with a trade imbalance}

Now, trade imbalance is introduced thorugh an export and import flow.
There are two countrie in the economy, namely country 1 and country 2.
The economy of the country $i$, with $i = 1,2$ is defined as in the previous
subsection, with the additional restriction that 
$$
E_t^1 + E_t^2 = 0,
$$
for every $t \geq 0$.
Given the interest rates of the countries, the economy is completely characterized
by two MFGs.

The price level may be identical, or may be regulated. Also, the production
functions $F^1$ and $F^2$ can be different, reflecting the existence of distinct
technologies in those countries. 

Each country's economy has an associated MFG system. Each of the central
banks controls an interest rate $r_t^i$ and seeks to maximize
$$
\sup_{r_t^i} \int^\infty_t e^{-\beta (s - t)} U(A_s^i, W_s^i, p_s^i, E_s^i) ds,
$$
where the state of the system is described by their MFGs together with their
assets dynamics
$\dot A^i_t = - r_t^i \int a^i \rho(aî, k^i, t) da\, dk.$

Therefore, this formulation can be regarded as non-cooperative game for two 
macroeconomic agents. Their states evolve through the asset dynamics and are restricted
by the MFGs. Note that the MFG of both countries are coupler through $E_t^1 + E_t^2 = 0$.



\end{document}
