\documentclass{article}
\usepackage[utf8]{inputenc}
\usepackage{amsmath}
\usepackage{amssymb}
\usepackage{amsthm}
\usepackage{amscd}
\usepackage{amsfonts}
\usepackage{mathtools}
\usepackage{amsmath}
\usepackage{amssymb}
\usepackage{amsthm}
\usepackage{amscd}
\usepackage{amsfonts}
\usepackage{mathtools}
\usepackage{xcolor}
\usepackage[
backend=biber,
style=alphabetic,
sorting=ynt
]{biblatex}


\newtheorem{proposition}{Proposition}
\newtheorem{remark}{Remark}


\newcommand{\RR}[0]{\mathbb{R}}
\newcommand{\EE}[0]{\mathbb{E}}
\newcommand{\bx}[0]{{\bf x}}
\newcommand{\bv}[0]{{\bf v}}
\newcommand{\balpha}[0]{{ \bf\alpha}}
\newcommand{\HH}{\mathcal{H}}
\newcommand{\heff}{\bar h^{\text{eff}}}


\addbibresource{sample.bib}

\newcommand{\bx}[0]{{\bf x}}
\newcommand{\bv}[0]{{\bf v}}


\newtheorem{proposition}{Proposition}
\newtheorem{remark}{Remark}

\newcommand{\HamiltonJacobiPDE}[2]{
%#1: Value function, #2: independent running cost
#1_t (x,t) + H(x,D_x #1(x,t)) = #2
}
\newcommand{\HamiltonJacobiTerminalCondition}[2]{
%#1: Value function, #2: Terminal payoff
#1(x,T) = #2(x)
}


\newcommand{\DensityTransportPDE}[2]{
    % #1: Value function, #2: Density
    #2_t(x,t) + \Delta(H_p(x, D_x #1 (x,t)) #2(x,t)) = 0
}

\newcommand{\DensityTransportInitialCondition}[1]{
    % #1: Density
    #1(x,t_0) = #1_0(x)
}


\newcommand{\ReducedMfgStateEvolution}[2]{
\begin{equation}\label{reduced_mfg:state_evolution}
\begin{cases}
    \dot{\bf #1}_t = {\bf #2}_t,\quad t\in (t_0,T],\\
    {\bf #1}_{t_0} = #1.
\end{cases}
\end{equation}
}

\newcommand{\ReducedMfgPayoffFunctional}[5]{
\begin{equation}\label{reduced_mfg:payoff_functional}
J({\bf #2},#1; t_0) = \int_{t_0}^T \left(#3({\bf #2}_s,{\bf #1}_s) + #5 \right) ds + #4({\bf #1}_T),
\end{equation}}

\newcommand{\ReducedMfgHamiltonian}[4]{
\begin{equation}\label{reduced_mfg:HJ}
H(#1,#2) \coloneqq \sup_{#3} ( #2 \cdot #3 + #4(#3,#1)).
\end{equation}
}


\newcommand{\ReducedMfgStateHamiltonJacobiTerminalValueProblem}[3]{
\begin{equation}\label{reduced_mfg:state_evolution_hamiltonian}
\begin{cases}
    \HamiltonJacobiPDE{#1}{#2}, \\
    \HamiltonJacobiTerminalCondition{#1}{#3}.
\end{cases}
\end{equation}}

\newcommand{\ReducedMfgFeedbackStateEvolution}[3]{
\begin{equation}\label{reduced_mfg:feedback_state_evolution}
\begin{cases}
    {\dot {\bf #1}}_#2 = H_p({\bf #1}, D_x #3({\bf #1},#2)),\\
    {\bf #1}_{t_0} = #1.
\end{cases}
\end{equation}}


\newcommand{\ReducedMfgDensityTransportInitialValueProblem}[2]{
\begin{equation}\label{reduced_mfg:density_transport_pde}
\begin{cases}
    \DensityTransportPDE{#1}{#2},\\
    \DensityTransportInitialCondition{#2}
\end{cases}
\end{equation}}

\newcommand{\ReducedMfgMeanFieldGameSystem}[4]{
\begin{equation}\label{reduced_mfg:mean_field_game_pde}
\begin{cases}
    \HamiltonJacobiPDE{#1}{#4},\\
    \DensityTransportPDE{#1}{#2},\\
\end{cases}
\end{equation}}

\newcommand{\ReducedMfgMeanFieldGameBoundaryConditions}[4]{
\begin{equation}\label{reduced_mfg:mean_field_game_boundary}
\begin{cases}
    \HamiltonJacobiTerminalCondition{#1}{#3},\\
    \DensityTransportInitialCondition{#2}.
\end{cases}
\end{equation}}



\title{Economic Models and Mean-Field Games Theory - Summary}
\author{Felipe Antunes}
\date{August 2022}

\begin{document}

\maketitle
\section{Introduction}
Mathematical methods allow for testable models in modern economic theory. It enables authors to formulate and solve models in a unified language.

Mean-field games - Methods and techniques to study differential games with a large population of rational players. Typically formulated in terms of partial differential equations, namely a transport or Fokker-Plank equation for the distribution coupled with a Hamilton-Jacobi equation.

Applications in economics, sustainable development.

The importance with regards to economic theory stems from the fact that it allows a systematic formalization ogf rational expectations theory and heterogeneous agent models.

Existence and regularity of solutions is an important research area.


\section{Deterministic Models}

\subsection{First-order mean-field games}

\subsubsection{Deterministic optimal control problems}\label{deterministic_optimal_control}

Single agent whose state is denoted by $x_t \in \RR^d$, for $t_0 \leq t\leq T$, where $T > 0$ is arbitrarily fixed. The case $T< \infty$ is referred to as the finite horizon optimal control problem.

$$\dot x_t = f(v_t,x_t),$$

where $f: \RR^m \times \RR^d \mapsto \RR^d$ is a given function and $v_t \in \RR^m$ is a control. Let $\mathcal{W} \coloneqq \left\{ v : [t_0, T] \mapsto W, v \in X \right\}$, where $X$ is a suitable function space.

Agent's preferences are expressed through a utility functional $J(u_t, x_t; t)$. For $t \in [t_0,T)$ this functional is given by 

$$
J(u,x,t) = \int_t^T u(v_s, x_s) ds + \Psi(x_T),
$$
where $u : \RR^m\times \RR^d \mapsto \RR$ is the instantaneous utility (or running cost) and $\Psi : \RR^d \mapsto \RR$ is the terminal payoff.

In the case $T = \infty$, we introduce a discount rate $\beta > 0$ and consider
$$
J_\beta (u,x;t) = \int_t^\infty e^{-\beta (s - t) u(v_s, x_s)} ds 
$$

The focus of this section will be a variant of the last problem, namely:

$$
J_\alpha (u,x;t) = \int_t^\infty e^{-\alpha (s - t) u(v_s, x_s)} ds + e^{-\alpha (T -t)} \Psi(x_T) 
$$

The single agent seeks to maximize $J_\alpha$ over all possible controls.
We define the value function $v(x,t)$ as

$$
V(x,t) = \sup_{v\in \mathcal{W}} J_\alpha (v, x ; t).
$$
This function is the maximum possible payoff for the agent at state $x$ in time $t$.

Assumptions on $f$ (for the problem to be addressed rigorously):
$$f \in C(\RR^m \times \RR^d, \RR^d),\, |f(v, x_1) - f(v, x_2)| \leq C_\theta |x_1 - x_2|, x_1, x_2 \in \RR^d, v\in \RR^m, |v| \leq \theta$$

If the Value function is of class $C^1(\RR^d \times [t_0,T])$, it solves  a nonlinear partial differential equation known as the Hamilton-Jacobi equation:
$$V_t(x,t) - \alpha V(x,t) + H(x,D_x V(x,t)) = 0,$$
where the Hamiltonian H is defined as 
$$H(x,p) = \sup_{v\in W} [p \cdot f(v,x) + u(v,x)],$$
that is, the Legendre transform of the running cost $u$.
The optimal control $v_t^*$ is given by 
$$ H_p (x^*, D_x V(x^*,t)) = f(v^*, x^*)$$

\subsubsection{Dynamic Programming Principle}

\begin{proposition}[Dynamic programming principle]
    Let $r \in (t_0, T)$ be fixed and assume that $V$ is a value function. Then we have
    $$
    V(x,t) = \sup_{v\in \mathcal{W}} \left( \int_r^T e^{-\alpha (s-r)} u(v_s,x_s) ds + e^{-\alpha (r - t) V(x_r,r)} \right).
    $$
\end{proposition}

\begin{proof}
    To do
\end{proof}

\subsubsection{Verification Theorem for Hamilton-Jacobi}
\begin{proposition}[Verification Theorem]
    Let $U \in C^1(\RR^d \times [t_0, T])$ be a solution to 
    $$
    U_t(x,t) - \alpha U(x,t) + H(x, U_x(x,t)) = 0\, \text{(HJ)}
    $$
    where 
    $$
    H(x,p) = \sup_{v \in W} [p \cdot f(v,x) + u(v,x)],
    $$
    with terminal condition $U(x,T) = \psi(x)$. Then
    $$U(x,t) \geq V(x,t), \forall (x,t) \in \RR^d \times [t_0,T].$$
    Moreover, if there exists a control $v^*$ such that
    $$
    H(x^*_t, U_x (x^*_t, t)) = U_x(x^*_t, t)\cdot f(v^*_t, x^*_t) + u(v^*_t, x^*_t),
    $$
    That is, $v^*$ gives an equality in the Hamiltonian, then $v^*$ is an optimal control for the control system and $U(x,t) = V(x,t)$.
\end{proposition}

\begin{proof}
    To do
\end{proof}

\subsubsection{Pontryagin Maximum Principle}

The PMP gives a necessary condition for an optimal control in terms of an adjoint variable $q: \RR \mapsto \RR^d$. The adjoint variable satisfies:
\begin{equation}\label{pmp:adjoint_eq}
    \frac{d}{ds} \left( e^{-\alpha(s-t)} {\dot q}_j(s) \right) = -e^{-\alpha (s - t)} \sum_{i = 1}^d \frac{\partial}{\partial x_j} f_i (v^*_s, x^*_s) q_i(s) - e^{-\alpha (s - t)} \frac{\partial}{\partial x_j} u(v^*_s, x^*_s),
\end{equation}

with $q(T) = \nabla \psi(x^*_T)$.

\begin{proposition}[Pontryagin Maximum Principle]
    Let $v^*$ be an optimal control for the optimal control problem and assume that $q$ is a solution for the terminal value problem \eqref{pmp:adjoint_eq}. Then,
    \begin{equation}\label{pmp:transversality_condition}
    q(s) \cdot f(v^*_s, x^*_s) + u(v^*_s, x^*_s) = H(x^*_s, q(s)).
    \end{equation}
\end{proposition}

\begin{proof}
    To do.
\end{proof}

Condition \eqref{pmp:transversality_condition} is called a \textit{tranversality condition}.

\subsection{Transport Equation}\label{transport_eq}
Consider a population that is governed at the single agent level by
\begin{equation}\label{transport_eq:ode}
\begin{cases}
    \dot x_t = b(x_t,t), \quad t > t_0,\\
    x_{t_0} = x.
\end{cases}
\end{equation}
where $b$ is a Lipschitz vector field. The vector field induces a flow that trasports the population's density along time (pushforward). The evolution of the density is described be the \textit{transport equation}, a first order pde:  

\begin{equation}\label{transport_eq:pde}
\rho_t + \Delta(b(x,t) \rho) = 0.
\end{equation}

\begin{proposition}[Properties of solutions]    
Let $\rho$ be a solution of \eqref{transport_eq:pde} with initial condition $\rho(x,t_0) = \rho_0(x)$, for some probability density $\rho_0 \in \mathcal{C}_c^\infty (\RR^d).$ Then we have:
\begin{itemize}
    \item Positivity of solutions: $\rho(x,t) \geq 0$ for all $t> t_0$,
    \item Mass conservation: $\int_{\RR^d} \rho(x,t) dx = \int_{\RR^d} \rho_0(x) dx = 1$ for all $t > 0$.
\end{itemize}
\end{proposition}

\begin{proof}
    To do.
\end{proof}

Define the flow in $\RR^d$ given by $b$ by $\Phi_t$. This flow maps $x\in \RR^d$ to the solution of \eqref{transport_eq:ode} at the instant $t > t_0$, with initial condition $x$.

We can use the operator $\Phi_t$ to define a family of measures $\theta(x,t)$ over $\RR^d$ from an initial measure $\theta_0$ by 
\begin{equation}\label{transport_eq:family_of_measures}
\int_{\RR^d} \phi(x) \theta(x,t) dx = \int_{\RR^d} \phi(\Phi_t(x)) \theta_0 dx    
\end{equation}

This measure satisfies the transport PDE in the distribution sense. 
\begin{proposition}[Solutions in the distribution sense]
Let $\theta$ be the measure defined in \eqref{transport_eq:family_of_measures}. Assume that the vector field $b$ is Lipschitz continuous and denote by $\Phi_t$ the flow corresponding to \eqref{transport_eq:ode}. Then
\begin{equation}\label{transport_eq:pde_for_measures}
    \begin{cases}
        \theta_t(x,t) + \Delta(b(x,t) \theta(x,t)) = 0, \quad (x,t) \in \RR^d \times [t_0,\infty),\\
        \theta(x,t_0) = \theta_0(x), \quad x\in \RR^d.
    \end{cases}
\end{equation}
in the distributional sense.
\end{proposition}

\begin{proof}
    To do.
\end{proof}

\begin{proposition}[Dirac delta solution]\label{transport_eq:prop:dirac_delta_solution}
Assume that \eqref{transport_eq:ode} holds and suppose that $b$ is Lipschitz continuous. Then, $\delta_{\Phi_t(x)}$ solves \eqref{transport_eq:pde_for_measures} in the distributional sense.
\end{proposition}

\begin{proof}
    To do.
\end{proof}

\begin{remark} The converse of \eqref{transport_eq:prop:dirac_delta_solution} also holds.
\end{remark}

\subsection{Reduced Mean-field games}\label{reduced_mfg}
Among the main motivations for studying MFG are the connections with the theory of $N$-player differential games. A solution of the MFG formalizes the limit case of Nash equilibrium when $N \to \infty$ for $N$-person differential games.

Some reduced mean-field games models can be rigorously derived as the limit of equations characterizing an $N$-player differential game. This subsection presents a heuristic derivation.

Consider first the single-agent optimal control problem. Suppose that the state of an agent at time $t$ is characterized by the vector $\bx_t \in \RR^d$. In addition, assume that the following differential equation describes the state $\bx_t$:
\ReducedMfgStateEvolution{x}{v}
where $\bv_t$ is a control and $x\in\RR^d$ is a given initial condition. This agent seeks to maximize the functional:
\ReducedMfgPayoffFunctional{x}{v}{u}{u_T}{g[\rho]}
where $u$ is a Lagrangian, $u_T$ is a terminal payoff and $g[\rho]$ is a term to be made precise later. 

Define
\ReducedMfgHamiltonian{x}{p}{v}{u}
We know from Section \eqref{deterministic_optimal_control} that under some regularity assumptions, the value function $V$ associated with the optimal control problem \eqref{deterministic_optimal_control:optimal_control_problem} 
solves the Hamilton-Jacobi equation in $\RR^d \times [t_0,T)$:
\ReducedMfgStateHamiltonJacobiTerminalValueProblem{V}{g[\rho]}{u_T}
Moreover, the optimal control $\bv^*$ is given in feedback form by
$$\bv^* = H_p(\bx, D_x V(\bx, t)).$$
The state of a rational agent will evolve according to \eqref{reduced_mfg:state_evolution} with the control $\bv$ equal to the feedback optimal control $\bv^*$.

Consider then a population of rational and indistinguishable agents. The distribution of agents is given by a probability measure $\rho$. This population of agents is subjected to the same optimal control problem. Therefore, the state of each individual is driven by
\ReducedMfgFeedbackStateEvolution{x}{s}{V}

The function $g$ encodes the dependence of the agent's cost functional on the density of the entire population. This implies that the optimization problem faced by the agents depends on the evolution of $\rho$. Conversely, the evolution of $\rho$ depends on a vector field determined by $V$, which is derived from the control problem of the agents.\footnote{Chicken and the egg problem.}

From Section \eqref{transport_eq}, we know that if $\bx$ satisfies \eqref{reduced_mfg:feedback_state_evolution}, then the evolution of the density $\rho$ in time satisfies a transport equation in $\RR^d \times (t_0, T]$:
\ReducedMfgDensityTransportInitialValueProblem{V}{\rho}

The mean-field game system associated with this problem is the system of PDE's:
\ReducedMfgMeanFieldGameSystem{V}{\rho}{u}{g[\rho]}
coupled with the boundary conditions
\ReducedMfgMeanFieldGameBoundaryConditions{V}{\rho}{u}{g[\rho]}

\section{Some Simple Economic Models}

\end{document}
